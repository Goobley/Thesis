\chapter{Time-Dependent Radiative Transfer}\label{Chap:TimeDepRt}

\setpythontexautoprint{false}
\begin{pycode}[TimeDepRT]
name = 'TimeDepRT'
chRT = texfigure.Manager(
    pytex,
    './02TimeDepRT',
    number=2,
    python_dir='./02TimeDepRT/python',
    fig_dir=   './02TimeDepRT/Figs',
    data_dir=  './02TimeDepRT/Data'
)
\end{pycode}

\begin{pycode}[TimeDepRT]
from lightweaver.fal import Falc82
from lightweaver.rh_atoms import H_6_atom, C_atom, O_atom, Si_atom, Al_atom, CaII_atom, Fe_atom, He_atom, MgII_atom, N_atom, Na_atom, S_atom
import lightweaver as lw
import matplotlib.pyplot as plt
import numpy as np

def iterate_ctx(ctx, Nscatter=3, NmaxIter=500):
    for i in range(NmaxIter):
        dJ = ctx.formal_sol_gamma_matrices()
        # NOTE(cmo): Do some initial iterations without touching the
        # populations to lambda iterate the background scattering terms
        if i < Nscatter:
            continue
        delta = ctx.stat_equil()

        # NOTE(cmo): Check convergence
        if dJ < 3e-3 and delta < 1e-3:
            print('Iterations taken: %d' % (i+1))
            print('-'*60)
            return

wave = np.linspace(853.9444, 854.9444, 1001)
def synth_8542(atmos, conserve, useNe):
    # NOTE(cmo): Configure the Gauss-Legendre angular quadrature for 5 rays
    atmos.quadrature(5)

    # NOTE(cmo): Construct the RadiativeSet with the following atomic models
    aSet = lw.RadiativeSet([H_6_atom(), C_atom(), O_atom(), Si_atom(), Al_atom(), CaII_atom(),
                            Fe_atom(), He_atom(), MgII_atom(), N_atom(), Na_atom(), S_atom()])
    # NOTE(cmo): Set Hydrogen and Calcium to active
    aSet.set_active('H', 'Ca')
    # NOTE(cmo): Compute the SpectrumConfiguration for this RadiativeSet
    spect = aSet.compute_wavelength_grid()

    # NOTE(cmo): If we're using the electron density provided with FAL C, then
    # compute the associated LTE populations, otherwise find a solution for
    # self consistent LTE populations and electron density.
    if useNe:
        eqPops = aSet.compute_eq_pops(atmos)
    else:
        eqPops = aSet.iterate_lte_ne_eq_pops(atmos)

    # NOTE(cmo): Construct the Context, optionally setting chargeConservation and the number of threads to use.
    ctx = lw.Context(atmos, spect, eqPops, conserveCharge=conserve, Nthreads=8)

    # NOTE(cmo): Iterate the NLTE problem to convergence
    iterate_ctx(ctx)
    # NOTE(cmo): Compute a detailed solution to Ca II 8542 on the 1 nm wavelength grid above
    Iwave = ctx.compute_rays(wave, [1.0], stokes=False)
    return Iwave

# NOTE(cmo): Load an atmosphere. In this case we include a copy of FAL C, but
# Lightweaver also supports loading atmospheres in the MULTI format, and it is
# also simple to do so from the raw data components
atmosRef = Falc82()
# NOTE(cmo): Ca II 8542 with the reference electron density in the FAL C atmosphere
IwaveRef = synth_8542(atmosRef, conserve=False, useNe=True)

atmosCons = Falc82()
# NOTE(cmo): Ca II 8542 with the electron density obtained from charge conservation
IwaveCons = synth_8542(atmosCons, conserve=True, useNe=False)

atmosLte = Falc82()
# NOTE(cmo): Ca II 8542 with LTE electron density
IwaveLte = synth_8542(atmosLte, conserve=False, useNe=False)

fig = plt.figure()
plt.plot(wave, IwaveRef, label='Reference')
plt.plot(wave, IwaveCons, '--', label='Conserved')
plt.plot(wave, IwaveLte, '--', label='LTE')
plt.title(r'Comparison of Ca\,\textsc{ii} 854.2\,nm with different electron densities', y=1.04)
plt.ylabel(r'Intensity [SI]')
plt.xlabel(r'$\lambda$ [nm]')
plt.legend()
lFig = chRT.save_figure('EleDensComparison', fig, fext='.pgf')
lFig.caption = r'Comparison of Ca\,\textsc{ii} 854.2\,nm with different electron densities'
\end{pycode}
\setpythontexautoprint{false}

\py[TimeDepRT]|chRT.get_figure('EleDensComparison')|

\begin{itemize}
    \item Current problems not solved by RADYN and MS\_RADYN.
    \item \Lw{}
    \item \MsLw{}
    \item Reprocessing RADYN simulations
    \item Effect of the Lyman lines on \CaLine{} in RHD simualtions
    \item Diagnostic potential of response functions
    \item Another item
\end{itemize}