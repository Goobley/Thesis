\chapter{Time-Dependent Radiative Transfer}\label{Chap:TimeDepRt}
%TC:group pycode 0 0
\setpythontexautoprint{false}
\begin{pycode}[TimeDepRT]
name = 'TimeDepRT'
chRT = texfigure.Manager(
    pytex,
    './02TimeDepRT',
    number=2,
    python_dir='./02TimeDepRT/python',
    fig_dir=   './02TimeDepRT/Figs',
    data_dir=  './Data/02TimeDepRT'
)
\end{pycode}

% \begin{pycode}[TimeDepRT]
% from lightweaver.fal import Falc82
% from lightweaver.rh_atoms import H_6_atom, C_atom, O_atom, Si_atom, Al_atom, CaII_atom, Fe_atom, He_atom, MgII_atom, N_atom, Na_atom, S_atom
% import lightweaver as lw
% import matplotlib.pyplot as plt
% import numpy as np

% def iterate_ctx(ctx, Nscatter=3, NmaxIter=500):
%     for i in range(NmaxIter):
%         dJ = ctx.formal_sol_gamma_matrices()
%         # NOTE(cmo): Do some initial iterations without touching the
%         # populations to lambda iterate the background scattering terms
%         if i < Nscatter:
%             continue
%         delta = ctx.stat_equil()

%         # NOTE(cmo): Check convergence
%         if dJ < 3e-3 and delta < 1e-3:
%             print('Iterations taken: %d' % (i+1))
%             print('-'*60)
%             return

% wave = np.linspace(853.9444, 854.9444, 1001)
% def synth_8542(atmos, conserve, useNe):
%     # NOTE(cmo): Configure the Gauss-Legendre angular quadrature for 5 rays
%     atmos.quadrature(5)

%     # NOTE(cmo): Construct the RadiativeSet with the following atomic models
%     aSet = lw.RadiativeSet([H_6_atom(), C_atom(), O_atom(), Si_atom(), Al_atom(), CaII_atom(),
%                             Fe_atom(), He_atom(), MgII_atom(), N_atom(), Na_atom(), S_atom()])
%     # NOTE(cmo): Set Hydrogen and Calcium to active
%     aSet.set_active('H', 'Ca')
%     # NOTE(cmo): Compute the SpectrumConfiguration for this RadiativeSet
%     spect = aSet.compute_wavelength_grid()

%     # NOTE(cmo): If we're using the electron density provided with FAL C, then
%     # compute the associated LTE populations, otherwise find a solution for
%     # self consistent LTE populations and electron density.
%     if useNe:
%         eqPops = aSet.compute_eq_pops(atmos)
%     else:
%         eqPops = aSet.iterate_lte_ne_eq_pops(atmos)

%     # NOTE(cmo): Construct the Context, optionally setting chargeConservation and the number of threads to use.
%     ctx = lw.Context(atmos, spect, eqPops, conserveCharge=conserve, Nthreads=8)

%     # NOTE(cmo): Iterate the NLTE problem to convergence
%     iterate_ctx(ctx)
%     # NOTE(cmo): Compute a detailed solution to Ca II 8542 on the 1 nm wavelength grid above
%     Iwave = ctx.compute_rays(wave, [1.0], stokes=False)
%     return Iwave

% # NOTE(cmo): Load an atmosphere. In this case we include a copy of FAL C, but
% # Lightweaver also supports loading atmospheres in the MULTI format, and it is
% # also simple to do so from the raw data components
% atmosRef = Falc82()
% # NOTE(cmo): Ca II 8542 with the reference electron density in the FAL C atmosphere
% IwaveRef = synth_8542(atmosRef, conserve=False, useNe=True)

% atmosCons = Falc82()
% # NOTE(cmo): Ca II 8542 with the electron density obtained from charge conservation
% IwaveCons = synth_8542(atmosCons, conserve=True, useNe=False)

% atmosLte = Falc82()
% # NOTE(cmo): Ca II 8542 with LTE electron density
% IwaveLte = synth_8542(atmosLte, conserve=False, useNe=False)

% fig = plt.figure()
% plt.plot(wave, IwaveRef, label='Reference')
% plt.plot(wave, IwaveCons, '--', label='Conserved')
% plt.plot(wave, IwaveLte, '--', label='LTE')
% plt.title(r'Comparison of Ca\,\textsc{ii} 854.2\,nm with different electron densities', y=1.04)
% plt.ylabel(r'Intensity [SI]')
% plt.xlabel(r'$\lambda$ [nm]')
% plt.legend()
% lFig = chRT.save_figure('EleDensComparison', fig, fext='.pgf')
% lFig.caption = r'Comparison of Ca\,\textsc{ii} 854.2\,nm with different electron densities'
% \end{pycode}
% \setpythontexautoprint{false}

% \py[TimeDepRT]|chRT.get_figure('EleDensComparison')|

% \begin{itemize}
%     \item Current problems not solved by RADYN and MS\_RADYN.
%     \item \Lw{}
%     \item \MsLw{}
%     \item Reprocessing RADYN simulations
%     \item Effect of the Lyman lines on \CaLine{} in RHD simulations
%     \item Diagnostic potential of response functions
%     \item Another item - PRD is hard?
% \end{itemize}

Modern Radiation Hydrodynamic (RHD) codes are highly complex, and contain many specialised features as discussed in the previous chapter.
In the following discussion we will focus primarily on \Radyn{}, the most widely used code of its ilk, and how using additional tools can facilitate new avenues of investigation.

\TODO{Explicit vs implicit}

\TODO{2 subcomponents}

\section{A Brief Dissection of \Radyn{} and the Future of RHD Modelling}\label{Sec:RadynDissection}

\emph{This section is informed by my discussions with Mats Carlsson, experiences using \Radyn{}, and in-depth analysis of its source code. It represents my own conclusions from the synthesis of these.}

\Radyn{}'s design closely follows its radiative transfer lineage. Its direct predecessor is the MULTI radiative transfer code and many commonalities remain.
NLTE radiative transfer is solved on a per transition basis using an ALI method and linearisation of the resultant level population balance equations.
This method is designed to solve the problem of non-overlapping lines but including an underlying background continuum.
This linearisation approach was proven by \citet{SocasNavarro1997} to be effectively equivalent to that of preconditioning for non-overlapping transitions \citep{Rybicki1991} for pure radiative transfer problems in the statistical equilibrium case.
This approach also assumed the use of a local diagonal $\Lambda^*$ operator, as discussed in the previous chapter.
\Radyn{}, however, chooses to employ a pentadiagonal $\Lambda^*$ operator to make optimal use of matrix bandwidth necessary elsewhere in the program and obtain improved convergence as a result.
It is unlikely that this change in operator significantly affects the conclusions of \citet{SocasNavarro1997}, but affects the two methods will no long arrive at the exactly equivalent numerical formulations.

The advantage of the linearisation approach in \Radyn{} is the ability to directly couple other equations to the RTE and implicitly solve all of these simultaneously and self-consistently.
Taking for example the kinetic equilibrium equation \eqref{Eq:KinEq}, \Radyn{}'s method formulates this expression such that the corrections from both the advection and population transition terms are considered simultaneously.
This is achieved through the use of a Newton-Raphson method, where the Jacobian is computed based on an analytic derivation, including the aforementioned linearisation of the kinetic equilibrium equation.
This same process simultaneously solves for the heat conduction, hydrodynamics of the system, and the new locations of the dynamic grid of \citet{Dorfi1987}.

A significant benefit of this implicit approach is a relaxation of the timestep constraints present in explicit approaches.
This is particularly important when considering the very fine grid spacing often required by the dynamic grid, which combined with the large bulk velocities occurring in flares can lead to extremely oppressive timestep constraints.

Despite its elegance, there are several major downsides to this implicit approach.
Foremost of these is the complexity engendered by the coupled design of the system, and the need to ensure that all necessary derivatives are analytically computed and included.
This presents a very large barrier to entry for future developments on the platform, and this is likely part of the reason why both Fokker-Planck modules integrated in \Radyn{} have operated externally to this core coupled system.
This overlapping of concerns
Additionally, implicit codes, whilst having reduced timestep constraints, are typically much more costly per timestep than explicit codes.
This is somewhat offset by the majority of the cost of each step residing within the formal solver, which remains similar in both cases.
The dynamic grid can also become problematic due its lack of interpretability, and propensity to drop to spacings finer than the local cyclotron radius in more energetic simulations.

\Radyn{} is a fantastic tool that has enabled insight into many different flare associated phenomena, and through these comments we do not intend to discredit its use, but instead highlight avenues for future development within the field of RHD.
As the different uses of \Radyn{} continue to evolve in complexity, with projects such as multi-strand arcade and minority species modelling \NeedRef{}, the code at the core of \Radyn{} will need to be modified by different researchers, and work facilitating this and highlighting additional factors to be considered in RHD modelling is core to the future development of this field.

As discussed previously in respect to \Lw{} and radiative transfer, the flexibility of a framework designed for solving a class of problem can yield significant advances in productivity.
The task of designing, constructing, and testing such a framework for a problem such as the complete quasi-one-dimensional RHD simulation flares is too significant to be undertaken here.
Nevertheless, it may prove a convenient future development once the necessary specifications are defined.
Here, we focus on reprocessing aspects of the radiative transfer of previously computed \Radyn{} simulations and investigate important directions for future developments in RHD modelling of flares.

\section{Minority Species}

For flares, \Radyn{}'s primary focus is on the major spectral lines and continua of hydrogen, helium, and calcium.
These typically represent the bulk of the radiative energy lost in the chromosphere.
Singly ionised magnesium has also been shown to be an important contributor to these energy losses, however the h and k lines require a treatment including PRD to avoid significantly overestimating their losses.
The \Caii{} H and K lines are also somewhat affected by PRD, in addition to the hydrogen Lyman lines.
For the Lyman lines we will discuss several strategies for approximating this treatment, and for \Caii{} H and K, it has been suggested that considering these in CRD approximately accounts for the lack of Mg\,\textsc{ii} h and k if all of these transitions were treated with PRD \citep{Kerr2019a}.

Whilst the lines of these four species are some of the strongest in solar spectrum, and their continua mediate much of the energy leaving the chromosphere, there are other chromospheric transitions that can also be used to diagnose the atmosphere.
For example, Si\,\textsc{iv} optical thickness has been investigated in a minority species context by \citet{Kerr2019c}.
An element treated as a ``minority species'' is assumed to not interact significantly with the energy balance of the simulation (i.e. the thermodynamic response of the model does not change significantly if this species is subject to a complete radiative treatment).
This should be true for most species with trace populations.
The radiative transfer calculations associated with this species can then be performed in a ``second-pass'' over a previously computed \Radyn{} simulation.
The MS\_RADYN code was designed for this task; it takes the thermodynamic parameters from every timestep of a \Radyn{} simulation, along with the non-equilibrium hydrogen populations, and solves the kinetic equilibrium equation at each timestep for a minority species.
Due to the lack of atmospheric thermodynamic response to changes in the radiative output of this species, far more complex atomic models can be used, such as the 30 level model silicon atom used by \citet{Kerr2019c}.

An approach similar to that of minority species modelling can be applied to testing the methods used in \Radyn{} and the importance of certain omissions.
Due to the reduced complexity of solving the kinetic equilibrium equations rather than the entire RHD system, these calculations typically run significantly faster than the original simulation.
In the following we will discuss the creation of a minority species tool for reprocessing \Radyn{} simulations, built on the \Lw{} framework, as well as its application to investigating the importance of overlapping transitions and discussing the difficulties of including PRD in these simulations.
From the previous discussion, building such a tool on the \Lw{} framework should provide researchers with a modern, simpler codebase that is easier to conceptualise and modify, allowing for investigation of effects to be included in \Radyn{} or future RHD codes.
Excluding the model atom definitions, the source code of the simulations presented in this chapter totals $\sim$1000 lines of Python, mostly following modern best practices.

\section{Reprocessing \Radyn{} Simulations with the \Lw{} Framework}

To perform a minority species simulation, a particular file from the original simulation, \texttt{atmost.dat}, must be provided.
From investigating the contents of this file we can determine the exact configuration of \Lw{} and the equations to be solved.
This file is written to for every internal timestep of the \Radyn{} simulation, and represents a limited subset of the less frequently written ``complete'' output (typically stored every \SI{0.1}{\second}).
It contains some metadata describing the size of the simulation, then for each internal timestep, the current timestep, the elapsed time at each timestep, the current locations of the dynamic grid, the mass density profile, the electron density profile, the temperature structure, the vertical velocity, and the current hydrogen level populations.

For the validation of \Lw{} and this style of simulation, we also wished to compute and compare the hydrogen populations to those computed in \Radyn{}.
Several difficulties arose due to the non-thermal collisional rates used in the kinetic equilibrium calculation for hydrogen and helium.
The non-thermal collisional rates of \citet{1993Fang} are used to determine hydrogen ionisation, and require the beam energy deposition throughout the atmosphere at each timestep.
For helium, if the Fokker-Planck electron beam description is used, then the rates of \citet{Arnaud1985} are used, but these require integration over the electron energy distribution.
Whilst it is possible to add both of these to the \texttt{atmost.dat} file, the complete electron distribution information is very large, and we instead elect to use ``Emslie'' beam electron formalism \citep{Emslie1978}, for which the energy deposition profile throughout the atmosphere is sufficient to describe the non-thermal rates.
We therefore chose to slightly modify \Radyn{} and add the beam heating profile to the \texttt{atmost.dat} file.
Our function for reading these files can handle both files with and without this modification.
In the event that this beam heating information is not saved, an approximation of it can be reconstructed via interpolation from the information in \Radyn{}'s complete save file.
Our testing of this show that the approximation is relatively good, but short term, or particularly narrow heating features may be lost.
For this reason, all simulations presented here use the version with this data.

With the above data we have sufficient information to construct the \Radyn{} thermodynamic atmosphere at any of its internal timesteps.
\Lw{} does not make use of the mass density directly, but instead maps it to hydrogen density.
For this we use of the default abundances in \Lw{}, based on \citet{Asplund2009}.
These differ to those used in \Radyn{}, but not significantly for any of the species discussed here.
Ignoring the advection term it is then simple to produce a minority species tool using this approach.
In many situations, the advection term has a small effect, and can safely be ignored \NeedRef{} \TODO{Old Flarix paper}.
The radiative transfer equation employed in \Radyn{} is formulated on the dynamic grid, and this is not the case in \Lw{}, which assumes that the grid is static (although this limitation can be worked around).
We can simply use a fixed denser spatial sampling of the atmosphere to account for the motion of features such as the transition region over the course of the simulation.
This model then interpolates the thermodynamic properties and NLTE hydrogen populations for the starting atmosphere onto our stratification and computes the statistical equilibrium solution for the minority species in question.
For each subsequent timestep these properties are interpolated from the new \Radyn{} grid to the static grid, and the minority species populations can be advanced in time using the process described in Sec.~\NeedRef{}.

To reduce the number of grid points needed for a static stratification one could instead use a fixed column mass stratification.
The transition region moves very little in terms of column mass during the simulation, however it then becomes necessary to interpolate the populations from one column mass stratification to the next; a process which can introduce significant error if not undertaken with care.
This error can be reduced by renormalising each specie's total number density throughout the atmosphere from the mass density and abundances, and this helps to avoid errors growing around regions of high gradient, such as the transition region.

To solve the minority species problem properly in a manner compatible with \Radyn{} it is necessary to include the advection terms.
In the following we will describe several different approaches to handling these advection terms, and thus solving the complete kinetic equilibrium equations.

\subsection{Advection}

In Sec.~\ref{Sec:RadynDissection} we discussed the coupled nature in which \Radyn{} solves the RHD equations.
For flexibility we wish to decouple the advection terms from the population transitions.
Our initial approach was to use an explicit method on the dense, fixed grid discussed above.
Explicit methods are typically simpler to validate and understand, so seemed a logical fit for this project.

\TODO{Describe WENO-HD -shocks etc}

To improve the agreement with \Radyn{}, reduce the number of interpolations,. and leverage its adaptive grid we instead decided to apply its technique for advection, but keep it distinct from the radiative transfer.
Thus, at the start of each timesteps we advect the populations from the previous grid to the new grid locations, and then advance the populations in time based on the NLTE rates.
This technique is known as operator splitting \NeedRef{}, and is commonly applied in hydrodynamics simulations where different numerical techniques are effective at solving different components of the system.
Performing a na\"{i}ve splitting like this limits the accuracy of the solution to first order accuracy in time.
Various advanced splitting techniques, such as Strang splitting \citep{Strang1968}, have been developed to reduced this error.
Strang splitting reduces these error terms to second order, and newer approaches can reduce the impact of operator splitting to fourth order in time \NeedRef{}.
A Strang splitting approach was implemented in our code using these components, but the results were not found to differ appreciably.
It is likely that this is due to one of the two processes dominating.
During explosive heating the populations in the upper chromosphere typically have very large transition rates, and comparatively slow advection.
At later times, when the timestep is longer, the transition rate is typically much lower and the plasma flows are relatively smooth, but often high velocity.

As discussed above, it is necessary to accurately and conservatively resolve shocks in advection and hydrodynamics problems, something that has been recognised since the 1950s \NeedRef{} \TODO{FCT book?}.
For this \Radyn{} employs a variant of the second order spatially accurate method of \citet{VanLeer1979}.
The concept of this class of method is similar to that of short characteristics formal solvers, ensuring that no spurious minima or maxima are injected into the solution, which in hydrodynamics would result in a gain or loss of density.

\TODO{I think we need a clear introduction to shock capturing methods. Read over pyCLAW paper and Leveque intro.}