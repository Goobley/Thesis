\chapter{Numerical Flare Modelling}\label{Chap:FlareModelling}
% spell-checker: disable
%TC:group pycode 0 0
\begin{pycode}[FlareModelling]
name = 'FlareModelling'
chFlareModelling = texfigure.Manager(
    pytex,
    './01aFlareModelling',
    number=1,
    python_dir='./01aFlareModelling/python',
    fig_dir=   './01aFlareModelling/Figs',
    data_dir=  './Data/01aFlareModelling'
)
\end{pycode}
% spell-checker: enable

% \begin{itemize}
%     \item Radiative Transfer Physics
%     \item Hydro and Radiation hydrodynamics
%     \item Numerical approaches to RT and RHD. (combined into the previous sections)
% \end{itemize}

Simulation is a powerful scientific approach that seeks both to validate our understanding of a phenomenon and also learn how it is influenced by various parameters.
Contrary to its dictionary definition, which suggests deceit or merely a surface level resemblance, simulations are essential tools in an astrophysical context.
Astrophysical observations differ greatly from laboratory experiments, due to our lack of knowledge of the configuration and parameters of the system, and inability to repeat particular events.
Simulations therefore represent our best tool for bridging the gap between a theory of the physical processes at work in an observed event, and the observations thereof.
A numerical approach is often needed as the coupled physical processes investigated are typically too complex and non-linear to analytically solve in detail.
There will commonly be free parameters left in these models, some of which can be directly inferred from observation, but others may need to be investigated and later constrained by comparing the model output to observations.
Due to computational or conceptual limitations, simplifying assumptions are often required to make these models tractable.


% In the following we will describe the primary physical processes assumed to be at work in solar flares, and how these are modelled, including the primary approximations used.


% \section{Modelling Flares and Observables}

To design a model of solar flares it is first necessary to understand as much as possible from the available observations, so as to ascertain reasonable approximations and the most important processes to include.
By the same, it is necessary to understand which observables vary between flares, to be able to exploit their specific sensitivities and choose those that exhibit sufficient variations to describe the processes at work and distinguish between flares.
The following builds on many years of work by the solar physics community, and we start from the understanding that has slowly been accrued to present the current state of flare modelling, and an introduction to some of the important numerical theory.
A discussion of the flare observables we are focused on, as well as the techniques used to obtain these, is presented in Chap.~\ref{Chap:FlareObservations}.

\section{Radiation Hydrodynamics}

% In our discussion of radiative transfer thus far we have treated the thermodynamic structure of the atmosphere as something constant, however given the short timescales flares evolve over, we know that this is a poor approximation.
A complete numerical description of a solar flare would require considering the simulation of a large volume of plasma permeated by a strong magnetic field, using the equations of magnetohydrodynamics (MHD) to describe the motion of the fluid, and considering the propagation of energy through the volume.
Due to the complexity of this treatment, including obtaining sufficient numerical resolution throughout the model, this is not currently feasible.
It is reasonable to assume that the strength of the magnetic field effectively locks the fluid within flux tubes to the field lines.
This is then used to construct an approach known as field-aligned radiation hydrodynamic (RHD) modelling.
RHD modelling of flares starts from the assumption that the plasma to be modelled can be represented as being contained within a tube, often considered semi-circular.
This represents a single magnetic flux tube, and the plasma may only move longitudinally along it.
The atmosphere is then treated as a plasma (ranging from semi-ionised in the lower atmosphere to fully ionised in the corona) consisting of a single compressible fluid, obeying the equations of hydrodynamics and energy transport through this medium.

This gas dynamic system can be described by
\begin{equation}
    \begin{aligned}
    \frac{\partial \rho}{\partial t} + \frac{\partial \rho \varv}{\partial z} &= 0\\
    \frac{\partial \rho \varv}{\partial t} + \frac{\partial \rho \varv^2}{\partial z} - \rho g + \frac{\partial P}{\partial z} - \frac{\partial}{\partial z}\left(\mu \frac{\partial \varv}{\partial z}\right) &= 0\\
    \frac{\partial E}{\partial t} + \frac{\partial}{\partial z}\left( E\varv + P\varv - \kappa\frac{\partial T}{\partial z} \right)+ \mu\left(\frac{\partial \varv}{\partial z}\right)^2 -\rho \varv g + L - S &= 0,
    \end{aligned}\label{Eq:RhdEquations}
\end{equation}

where $\rho$ is the mass density of the gas, $\varv$ is the bulk velocity, $z$ is the spatial coordinate, $g$ is the local gravitational acceleration, $P$ is the pressure, $\mu$ is the coefficient of dynamic viscosity, $E$ is the total plasma energy, $L$ is the radiative loss term, and $S$ is all additional energy source terms, from radiative transfer and atmospheric heating.
In these models the plasma is typically assumed to behave as an ideal gas, and therefore uses the equation of state
$P = n_{\mathrm{tot}} k_B T$, where $n_{\mathrm{tot}}$ is the total particle number density, $k_B$ is Boltzmann's constant, and $T$ is the plasma temperature. Additionally, $\kappa$ is the temperature-dependent coefficient of heat conduction, and is discussed in detail in Sec.~\ref{Sec:NumericalConduction}.

All of the quasi-one dimensional codes solve a variant of the RHD equations presented in \eqref{Eq:RhdEquations}. The first of these describes mass continuity, the second conservation of momentum, and the third conservation of energy.
The nuance typically occurs in ensuring that all terms are solved in a self-consistent fashion and the treatment of the optically thick radiation source and sink terms appearing in the energy equation.

The first generation of hydrodynamic flare models \citet{Nagai1980,Mariska1982,McClymont1983} focused on capturing the gas dynamics of the flaring event, using simplified radiative treatments.
These early models were, in general, much more limited due to computational constraints. \citet{Nagai1980} and \citet{Mariska1982} both employ optically thin losses with ad hoc corrections to avoid over-cooling the lower atmosphere.
\citet{McClymont1983} employ an escape probability formalism for determining the transition rates of hydrogen, which is a faster, less precise method than detailed radiative transfer, but vastly more accurate than assuming the chromosphere to be optically thin (or an ad hoc modulation thereof).
Escape probability methods are unable to directly take radiative backwarming into account -- where radiation produced higher in the atmosphere is absorbed at a lower point -- which can be an important factor for strong solar optical transitions used to diagnose atmospheric properties.

The current generation of RHD codes includes RADYN \citep{Carlsson1992a,Carlsson1995,Carlsson1999, Allred2015}, FLARIX \citep{Varady2010,Heinzel2015}, HYDRAD \citep{Bradshaw2003, Bradshaw2013}, and HYDRO2GEN \citep{Druett2018,Druett2019}.
RADYN and FLARIX both apply detailed treatment of the radiative losses for certain chromospheric transitions, whilst HYDRAD uses partially precomputed radiative rates for hydrogen (based on the treatment of \citet{Sollum1999}) and radiative losses following the approximations of \citet{Carlsson2012}.
HYDRAD originally focused more on the investigation of non-equilibrium optically thin radiative losses in the corona, and less on the chromosphere, however, recent development has also moved in the direction of an improved chromospheric treatment.
It also differs in its use of a two-fluid model, treating electrons and ions as separate, but coupled, fluids.
In current applications this does not appear to provide significantly different results when applied to flare modelling.
HYDRO2GEN applies a different approach to computing the radiative transfer terms using the approximate L2 method of \citet{Ivanov1984}, and electron beam heating following \citet{Syrovatskii1972}.

RADYN is derived from the MULTI radiative transfer code \citep{Scharmer1985, Carlsson1986, Carlsson1992}, and applies the linearisation approach described therein to the entire RHD system, ensuring the self-consistency of all terms. The equations of RHD are solved in an implicit linearised form by Newton-Raphson iteration on the dynamic grid of \citet{Dorfi1987}.
RADYN was originally constructed to investigate the effects of waves propagating through the chromosphere, but was extended by \citet{Abbett1999} to perform some of the earliest simulations of the dynamic flares with detailed radiation treatment in a self-consistent fashion. The code has since been developed further to include improved energy transport treatments, primarily focusing on electron beam energy deposition \citep{Allred2005, Allred2015}.
Its techniques are discussed further in Sec.~\ref{Sec:RadynDissection}.

FLARIX, on the other hand, consists primarily of three separate modules. The hydrodynamics is solved by a variant of the NRL Solar Flux Tube Model described in \citet{Mariska1982,Mariska1989}, extended with a MALI module (considering non-overlapping spectral lines with constant background continuum as described in \citet{Rybicki1991}) for detailed radiation transfer, and a test-particle code for determining heating and electron deposition \citep{Varady2010, Heinzel2015}.

RADYN and FLARIX have recently been tested against each other and shown to provide remarkable agreement, despite their different heritage \citep{Kasparova2019}, and early tests also show good consistency with HYDRAD when its recent extensions are enabled\footnote{Initial comparisons undertaken following the International Space Science Institute meeting: ``Interrogating Field-Aligned Solar Flare Models: Comparing, Contrasting and Improving'' led by G.S. Kerr and V. Polito.}.

\section{An Eye to the Future: Radiative Magnetohydrodynamics}

Much scientific effort is being invested into the creation of three-dimensional radiative MHD (RMHD) models, which aim to accurately describe all of the physics of a flaring system through the use of large models on supercomputers. The two leading codes are \textit{Bifrost} \citep{Gudiksen2011}, and MURaM \citep{Rempel2009,Rempel2016}. MURaM was originally focused on photospheric magnetoconvection, but has been extended into the corona. It adopts a grey radiative transfer technique that treats all outgoing radiation using a wavelength averaged approach, but has successfully demonstrated the ability to drive self-consistent flare-like eruptions with time-dependent photospheric magnetograms as a boundary condition \citep{Cheung2019}.

\textit{Bifrost} is more focused on a detailed chromospheric treatment; it can apply different combinations of physics modules including multi-group opacities for wavelength dependent radiation transport, the \citet{Sollum1999} model for hydrogen ionisation \citep[following the treatment of][]{Leenaarts2007}, and radiative losses following the method of \citet{Carlsson2012}.
It has been used to investigate solar enhanced networks \citep{Carlsson2016}, the generation of spicules \citep{Martinez-Sykora2017}, and the magnitude of magnetic energy present in the chromosphere \citep{Martinez-Sykora2019}, amongst many other projects.

Currently, neither code can handle the scale of the energy deposition used in the field-aligned flare models, or provide an equivalent chromospheric spatial resolution, but this is an area of very active development.
The field-aligned models remain complementary to these much more complex and computationally intensive models, allowing for more rapid investigation of the relative importance of phenomena to be implemented within the RMHD models, the use wider parameter spaces and higher energy inputs, and easier more compact output that is easier to manipulate and interpret.


\section{Solar Flare Heating}

The aforementioned modern field-aligned RHD codes are primarily designed to simulate the response of a tube of initially quiet solar atmosphere to heating.
This is typically presumed to be due to a beam of energetic electrons precipitating from the corona due to magnetic reconnection.

The direct effects of these electrons must also be considered, as their flux is sufficiently large that their collisions with particles in the lower atmosphere can substantially change the distribution of these particles, exciting and ionising them.
The evolution of the atmosphere will also affect how and where it interacts with these precipitating electrons, coupling an additional problem to the extant RHD system.

There are different methods of solving this problem, the simplest of which is the analytic solution of \citet{Emslie1978}, known as the ``Emslie beam'', which provides an analytic solution to electron beam energy deposition along the column of plasma under the assumption that all electron energy is lost due to Coulomb collisions with the plasma acting as a cold target.
This is a useful approximate treatment, however terms such as relativistic effects on high energy electrons, return current (heating the corona), magnetic mirroring, and particle diffusion (pitch-angle and momentum) are all important here, and ignored in this model. RADYN and FLARIX both include more advanced treatments including these effects; FLARIX uses a test-particle module \citep{Varady2010}, whereas RADYN solves the Fokker-Planck equation \citep[now using the method of ][]{Allred2020}, but both also include the option to use the the simpler analytic Emslie formulation.

The Alfvénic heating model has also been considered in modelling efforts.
A simplified variant of this has been investigated by \citet{Kerr2016} inside the \Radyn{} code, and produced notably different spectral line profiles which may agree better with observations than those from an electron beam model.
Further investigation of a more complete implementation of this method and its results are needed.
This approach was also incorporated into HYDRAD by \citet{Reep2016}, which showed the viability of these waves heating the deep chromosphere and triggering explosive chromospheric evaporation.
It is non-trivial to incorporate a full treatment of this phenomenon into RHD codes as it is coupled to the magnetic field, and future improvements are needed to more accurately model these effects and investigate to what extent Alfvénic and electron beam heating occurs simultaneously.

The simulations of flares discussed thus far allow us to model the time-evolution of heating a starting atmosphere and in turn predict line and continuum emission through the tools of radiative transfer.
These detailed results can be compared against observations, and have provided significant insight into chromospheric properties (e.g. \citet{Kuridze2015,RubioDaCosta2016,Kowalski2017,Simoes2017}).
The manual forward-fitting process involved in attempting to reproduce these observations with simulations is both time-consuming and difficult and lies close to the field of automated inversions, which we will address later in Chapters~\ref{Chap:FlareObservations} and \ref{Chap:Radynversion}.

We will now present a more detailed overview of the physics involved in the most important terms of the RHD equations \eqref{Eq:RhdEquations}, starting with complexities of radiative transfer outside of local thermodynamic equilibrium.

\section{Introduction to Radiative Transfer}\label{Sec:IntroRT}

\emph{The content of this section draws primarily from \citet{Hubeny2014} and the paper describing the \Lw{} radiative transfer framework \citep{Osborne2021}.}

Radiative transfer is the science of how radiation propagates through a material: the absorptions, scatterings and emission that happen therein.
Radiation is by far the most widely exploited conduit through which information can arrive from celestial bodies.
It allows us to derive proxies for \emph{in situ} measurements, that cannot otherwise be obtained, due to the distances and extreme conditions being observed.

Everyone is familiar with the concept of images, which show the spatial variation of light, but a lot of additional information can be gleaned by analysing radiation in terms of its wavelength variation and polarisation projections.
Using both of these properties is referred to as \emph{spectropolarimetry}, and \emph{spectroscopy} when only the unpolarised intensity is considered.
The most basic property to consider here is the specific intensity, commonly denoted $I(\nu, \vec{d})$ for a particular frequency $\nu$ and direction $\vec{d}$ at a location in space with typical SI units \si{\joule\per\square\metre\per\s\per\hertz\per\steradian}.

A ray propagating through a medium, such as a neutral gas or plasma, will gain a certain amount of energy per unit length due to emission processes, whilst also losing another amount due to absorption and scattering processes. These will depend on the local parameters of the plasma as well as the direction of the ray. For a plasma where the primary interacting species are atomic, we can distinguish bound-bound (spectral lines) and bound-free (continuum) transitions. In the former a bound electron moves between two different sublevels of an atom\footnote{Here \emph{atom} refers to either a neutral atom or an ion}, whilst in the latter the atom either absorbs sufficient energy to free a previously bound electron, or an ion loses sufficient energy to recombine with a free electron.

In addition to the obvious spontaneous emission and absorption processes, there is also a process of stimulated emission which is needed to balance the transitions. This occurs when an electron is stimulated to transition between levels by photons with the same frequency and direction as the photon produced by this transition.

In the following, we will discuss how to obtain the frequency- and direction-dependent outgoing radiation given the emissivity and opacity of the plasma, and how to obtain a self-consistent radiation field and populations for atomic species outside of the approximations of local thermodynamic equilibrium.
We will also discuss how to determine convergence for the iterative processes used.

\subsection{The Formal Solution}

For a one-dimensional planar atmosphere the radiative transfer equation (RTE) for the specific intensity along a ray can be written as
\begin{equation}
    \frac{1}{c}\frac{\partial I(\nu, \vec{d})}{\partial t} + \mu \frac{\partial I(\nu, \vec{d})}{\partial z} = \eta(\nu, \vec{d}) - \chi(\nu, \vec{d})I(\nu, \vec{d}),
    \label{Eq:FullTimeDepRte}
\end{equation}
where $\eta$ is the plasma emissivity, $\chi$ is the plasma opacity, $z$ is the spatial coordinate of the stratification of the atmosphere, $t$ represents time, $c$ is the speed of light, and $\mu$ is the cosine of the ray's inclination to the surface normal of the plane-parallel configuration.
We consider that the light-crossing time for the propagation of light on a solar scale is small compared to the time evolution of both the atmosphere and our observations so we ignore the time-derivative term.
Defining the source function
\begin{equation}
S(\mu, \vec{d}) = \frac{\eta(\nu, \vec{d})}{\chi(\nu, \vec{d})},
\end{equation}
and the optical depth along a ray from the observer as the number of photon mean free paths along this segment
\begin{equation}
\tau(z, \nu, \vec{d}) = \int_z^{z_{\mathrm{obs}}}  \frac{\chi_\nu(z^\prime)}{\mu}\, dz^\prime,
\end{equation}
we can write the RTE as
\begin{equation}
    \frac{\partial I(\nu, \vec{d})}{\partial \tau(\nu, \vec{d})} = I(\nu, \vec{d}) - S(\nu, \vec{d}).
    \label{Eq:1DRte}
\end{equation}

Equation \eqref{Eq:1DRte} is a first-order linear differential equation and can be solved with the integrating factor $e^{-\tau(\nu, \vec{d})}$ giving the formal solution of the RTE:
\begin{equation}
I(\tau_0, \nu, \vec{d}) = I(\tau_1, \nu, \vec{d}) e^{-(\tau_1 - \tau_0)} + \int_{\tau_0}^{\tau_1}S(t_\nu, \nu, \vec{d})e^{-(t_\nu - \tau_0)}\, dt_\nu,
\label{Eq:IntegratedRte}
\end{equation}
for $\tau_0$ the optical depth at the observer, $t_\nu$ a dummy variable used for integration, and $\tau_1 > \tau_0$ along the line of sight.

The solution in equation \eqref{Eq:IntegratedRte} prescribes nothing about the form of the source function in the atmosphere, and assumes that it varies continuously.
A \emph{formal solver} is a numerical routine that computes the intensity in a discretised atmosphere by solving a form of the RTE \eqref{Eq:1DRte}.
There are approaches such as that of \citet{Feautrier1964} that involve casting the problem as a second order differential equation
for both an up-going and a down-going rays simultaneously, it cannot handle both Doppler shifts and overlapping lines, and for these reasons we will not discuss it further.
We shall instead focus on the so-called short-characteristic method consisting of solving the RTE directly between discrete points by prescribing a functional form for its variation between these points.

\subsection{Short-Characteristics Methods}\label{Sec:ShortChar}

If a functional form with an analytic integral is chosen for the variation of the source function between defined points then an atmosphere can be treated as a sum of analytic integrals. We shall consider the simplest useful functional form, a linear variation, as an illustrative example.
This approach was first presented by \citet{Olson1987}.

The RTE for one frequency and direction through a slab of a plane-parallel atmosphere (in the case of outgoing radiation (i.e. $\mu > 0$)) can be written
\begin{equation}
    I(\tau_0) = I(\tau_1) \exp(- |\tau_0 - \tau_1|) + \int_{\tau_0}^{\tau_1} S(t) \exp(-(t - \tau_0))\, dt,
    \label{Eq:ShortCharForm}
\end{equation}
with $t$ as a dummy integration variable.

Now, assuming a linear variation of $S$ with $\tau$ in this slab gives
\begin{equation}
    S(t) = S_{\tau_0} \frac{\tau_1-t}{\tau_1-\tau_0} + S_{\tau_1} \frac{t-\tau_0}{\tau_1-\tau_0},
\end{equation}
where $S_{\tau_i}$ indicates the value of the source function at optical depth $\tau_i$.
This can then be substituted into \eqref{Eq:ShortCharForm} (with $\Delta := \tau_1 - \tau_0$) giving
\begin{equation}
    I(\tau_0) = I(\tau_1) \exp(- |\tau_1 - \tau_0|) +
    \frac{\Delta - 1 + \exp(-\Delta)}{\Delta} S_{\tau_0} +
    \frac{1 - \exp(-\Delta) - \Delta\exp(-\Delta)}{\Delta} S_{\tau_1}.
\end{equation}
This expression can be applied recursively from one end of the atmospheric model (assuming that the incoming radiation field is known), to the other, to provide the emergent intensity.
It is this procedure we refer to as computing a formal solution, and this will typically need to be computed for multiple directions (angles to the surface normal in the plane-parallel case) to compute the the angle-averaged radiation field at each frequency and location in as model atmosphere using a weighted quadrature, such as the Gauss-Legendre quadrature.

In atmospheres with very well resolved spatial grids, this method works quite well, however whenever the true source function has positive curvature, the intensity is overestimated, and underestimated for negative curvature. These effects can become quite significant in more sparsely sampled atmospheres.

The short characteristics method can be improved by using higher order polynomial interpolants, however these can lead to spurious ringing artifacts, negatively affecting their precision. One commonly used robust formulation is the monotonic piecewise parabolic method of \citet{Auer1994}. This method assumes a parabolic variation of the source function across local three consecutive points (i.e. a three point stencil), but limits it to the value obtained from linear interpolation if the parabolic interpolant exits the range bounded by these three points.

Other interpolating functions can be used. For example, the cubic Bézier spline technique of \citet{DelaCruzRodriguez2013}, provides a higher order approximation in regions of smooth variation, and can be limited through the control points to prevent any ringing instability. A similar approach has been taken with the BESSER quadratic Bézier spline approach of \citet{Stepan2013}.
For the plane-parallel case, all of these methods can be derived analogously to the linear formal solver presented above.
In Sec.~\ref{Sec:2DFS} we describe the implementation of the BESSER method in a two-dimensional atmosphere.

\subsection{Other Formal Solvers}

\citet{Janett2018} proposes a novel approach to the formal solver,  using an optimised solver for the differing optical thickness in each slab. They show that this leads to substantial performance benefits whilst also being more numerically stable. They comment that whilst higher order formal solvers will theoretically converge better to the true result, due to the assumptions that are made in their derivation, this will only occur if the variation of the source function in relevant regions of the atmosphere is sufficiently smooth. The modern 3D RMHD simulations use relatively coarse spatial grids with large transients in atmospheric parameters that risk provoking instability in the higher order formal solvers, especially in the case of full Stokes radiative transfer.

As work continues on these higher resolution RMHD simulations an investigation into how each formal solver handles discontinuous parameters is needed.
This has been discussed by \citet{Steiner2016}, and the methods employed for reconstructing discontinuous parameters in numerical hydrodynamics (see Sec.~\ref{Sec:HydroReconstruction} for a description of these) may present a future avenue for more robust, accurate, and efficient formal solvers.
An accurate treatment of steep gradients and discontinuities is not just valuable to the RMHD models, but important to all radiative transfer modelling, as a reduction in the spatial resolution needed to correctly evaluate radiative transfer can lead to significant reductions in computational requirements, making more complex simulations possible.


\subsection{LTE vs NLTE}

With the understanding of how to design and implement formal solvers for the radiative transfer equation, we have the ability to compute the radiation leaving an atmosphere from the opacity and emissivity structure.
The values of these depend on the atomic populations, and their distribution across energy levels and ionisation states.

As stars like the Sun radiate energy into the cosmos, it is clear that they cannot be in total thermal equilibrium.
However, it is reasonable to suggest that if the atmosphere is sufficiently collisional then a form of local thermodynamic equilibrium (LTE) holds, whereby local parcels of the plasma are effectively in thermodynamic equilibrium such that Kirchhoff's laws of radiation hold.
When a species within a plasma is in LTE its population distribution can be computed using the Saha-Boltzmann equation.
This is the case for spectral lines that form in the photosphere, and the associated level populations, emissivity, and opacity can all be computed directly from local thermodynamic quantities\footnote{If the electron density is not known \textit{a priori} then an iteration scheme using the Saha-Boltzmann equation is necessary to determine consistent values of both the electron density and the atomic populations.}.

If the radiative rates instead dominate the total transitional rates for the species then the particle distribution can no longer be governed by purely local parameters and we enter the realm of non-LTE (NLTE) physics.
This occurs as we enter the chromosphere, where the collisional rates decrease, and the radiation field couples the atomic populations in different regions together.
The core focus of NLTE radiative transfer is to determine atomic populations consistent with the local thermodynamic parameters of the atmosphere and the non-local radiation field.
Until otherwise specified, we will consider that the electron density throughout the atmosphere is known and consider it part of the atmospheric model.

Expressing this mathematically, we write the total transition rate $P_{ij}$ per atom in level $i$ between levels $i$ and $j$ (with the convention that $i < j$) as
\begin{equation}
    P_{ij} = R_{ij} + C_{ij},
\end{equation}
where $R_{ij}$ is the rate of radiative transitions (due to absorption, spontaneous, and stimulated emission) and $C_{ij}$ is the rate of collisional transitions.
As the total population of each element must remain constant we can write the kinetic equilibrium equation (which can be derived from the Boltzmann equation)
\begin{equation}
    \frac{\partial n_l}{\partial t} + \nabla \cdot (n_l \vec{\varv}) = \sum_{l^\prime\neq l} (n_{l^\prime} P_{l^\prime l}) - n_l \sum_{l^\prime\neq l} P_{ll^\prime},
    \label{Eq:KinEq}
\end{equation}
where $n_l$ is the number density of atoms in level $l$ of the atomic species in question.

Equation \eqref{Eq:KinEq} is frequently simplified to the statistical equilibrium equation, by setting the left-hand side to 0 giving
\begin{equation}
\sum_{l^\prime\neq l} (n_{l^\prime} P_{l^\prime l}) - n_l \sum_{l^\prime\neq l} P_{ll^\prime} = 0.
\label{Eq:StatEq}
\end{equation}
In Chaps.~\ref{Chap:TimeDepRt} and \ref{Chap:2DRT} we will discuss when the full time-dependent solution need be considered over the simpler statistical equilibrium solution.
The former of these requires a history of the atomic populations in the atmosphere, typically from a time-evolving model such as an RHD simulation, whereas the latter associates a unique solution of atomic populations to a given atmosphere.\footnote{The uniqueness of the solution was proven by \citet{Rybicki1997} for the linear case of \eqref{Eq:StatEq}. I have found no equivalent proof for the non-linear system when a variable electron density is also considered.}
In the following, even when considering the time-dependent problem, we assume that the timescale over which each step in our numerical simulation is integrated is long compared to the light-crossing time for the regions where each transition is optically thick.
If this is not the case, then a different formulation of the formal solver will be required, so as to solve \eqref{Eq:FullTimeDepRte}.

\subsection{Collisional Rates}

There are many collisional processes by which electrons can transition between energy levels in a plasma, including ionisation and recombination processes.
These include excitation of ions by electrons, ionisation and excitation of neutral by electrons, excitation by protons and neutral hydrogen, as well as charge exchange with these species.
There are more advanced rate formulations that depend on complex functional forms derived from laboratory work and theoretical analysis, such as those of \citet{Burgess1983, Arnaud1985}.

In most cases, the collisional rates are assumed to only depend on local plasma properties, such as the temperature, total particle density, and electron density, with the particles locally following Maxwellian velocity distributions that connect the upwards and downwards collisional rates for a process through detailed balance i.e., $n_i^* C_{ij} = n_j^* C_{ji}$, where $n_i^*$ indicates the LTE population of level $i$.
This is not the case for non-thermal rates, such as those of \citet{1993Fang} which consider collisional excitation by non-thermal electrons from the electron beams used in RHD simulations, but are still computed from local parameters.

\subsection{Emissivity and Opacity}

To mathematically formulate the radiative rates that are so important to $P_{ij}$ in a NLTE context we must first formulate expressions for emissivity and opacity.
There are two forms of transition we need to consider for atomic radiative transfer, spectral lines, and continua.
In spectral lines, the three processes that need to be understood to develop a model for emissivity and opacity are spontaneous and stimulated emission, and absorption.
The magnitude of these terms are controlled by the Einstein coefficients $A_{ji}$, $B_{ij}$, and $B_{ji}$ respectively, which are related by the Einstein relations
\begin{equation}
    g_i B_{ij} = g_j B_{ji},
\end{equation}
where $g_i$ is the statistical weight of level $i$ and
\begin{equation}
    A_{ji} = \frac{2h\nu_{ij}^3}{c^2}B_{ji},
\end{equation}
where $h$ is Planck's constant, $c$ is the speed of light, and $\nu_{ij}$ is the rest frequency of the spectral line.

The rest frequency $\nu_{ij}$ of a spectral line is defined by
\begin{equation}
    \nu_{ij} = \frac{\Delta E_{ji}}{h},
\end{equation}
where $\Delta E_{ji}$ is the energy difference between levels $j$ and $i$.
Bound-bound transitions between states in a static plasma are not infinitely narrow, but are instead broadened by a number of factors such as natural broadening from uncertainty in the lifetime of the upper state, Doppler broadening due to random thermal motions in the plasma, and collisional broadening (e.g. van der Waals, and Stark broadening\footnote{van der Waals broadening is due to an interaction between an excited atom and the dipole it induces over a neutral atom, whilst Stark broadening is due to the interaction between an atom and a charged particle.}). %(linear Stark effect for hydrogen and quadratic for all species following the Weisskopf approximation of impact theory).}). N.B. Quadratic typically << 0.1x linear for H.
The net effect of these processes typically leads to spectral line profiles being modelled as a Voigt function (the convolution of a Gaussian and a Lorentzian).
The Gaussian terms are due to Doppler broadening, whilst the other terms are typically modelled as Lorentzians \citep[e.g.][]{Sutton1978}, although more accurate treatments of Stark broadening such as those employed \citet{Kowalski2017} are non-Lorentzian, and must be separately convolved with the Voigt profile produced from the previous terms.
The normalised line absorption profile then describes the probability of a photon with a certain energy being absorbed by the transition.

Typically it is assumed that the plasma is sufficiently collisional for elastic collisions to redistribute the electrons of an atom in an excited state over all sub-states of an energy level prior to emission.
If this is not the case then the frequency of the outgoing photon will be correlated with the frequency of the photon that excited the atom into this state.
Thus the spectral line will have an emission profile distinct from its absorption profile.
This effect is known as partial redistribution (PRD) and will be discussed in Sec.~\ref{Sec:Prd}.
The typical state of affairs, where there is sufficient redistribution from elastic collisions for these two processes to be uncorrelated is known as complete redistribution (CRD).
In the following, we will always express line emission processes through an emission profile for generality, even if they are treated as CRD.

Continua, or bound-free transitions, depend instead on photoionisation cross-sections.
In the case of hydrogenic ions these fall off with $1/\nu^3$ (for photons of frequency $\nu$) as the energy of the ionising photon increases away from the continuum edge.
The edge of a continuum is defined by the ionisation potential for an atom in the bound state of this transition, as only photons with an energy equal to or greater than this are capable of photoionising the element.
In general these cross-sections and their variations are determined from laboratory experiments and numerical solutions of the Schr\"{o}dinger equation.
The photoionisation cross-section describes how likely the atom is to interact with a photon of a particular energy through this bound-free transition.
Its partner processes (like spontaneous and stimulated emission for bound-bound transitions) are spontaneous and stimulated recombination, whereby an electron is captured by the atom and a photon is released with energy equal to the excess energy of the system.
These are related to the photoionisation cross-section and the local plasma parameters by the Milne relations.

There are also free-free interactions between particles (often known as \emph{bremsstrahlung}), where electrons and ions interact, and the electron gains or loses energy through absorption or emission of a photon.
This process also has a characteristic cross-section and is considered as part of the background opacities and emissivities.

Following the notation of \citet{Rybicki1992} and \citet{Uitenbroek2001} the emissivity $\eta$ and opacity $\chi$ for a transition can then written
\begin{align}
    \label{Eq:Emis}
    \eta_{ij} &= n_j U_{ji}(\nu, \vec{d}), \\
    \label{Eq:Opac}
    \chi_{ij} &= n_i V_{ij}(\nu, \vec{d}) - n_j V_{ji}(\nu, \vec{d}).
\end{align}

The $U$ and $V$ terms are defined for bound-bound and bound-free transitions as
\newlength{\WidestCase}
\settowidth{\WidestCase}{$n_e\Phi_{ij}(T)\left(\frac{2h\nu^3}{c^2}\right)e^{-h\nu/k_B T}\alpha_{ij}(\nu),$}
\begin{align}
    \label{Eq:Uji}
    U_{ji} =&
    \begin{cases}
        \frac{h\nu}{4\pi}A_{ji}\psi_{ij}(\nu, \vec{d}), & \textrm{bound-bound} \\
        n_e\Phi_{ij}(T)\left(\frac{2h\nu^3}{c^2}\right)e^{-h\nu/k_B T}\alpha_{ij}(\nu), & \textrm{bound-free},
    \end{cases}\\
%
    \label{Eq:Vij}
    V_{ij} =&
    \begin{cases}
        \makebox[\WidestCase][l]{$\frac{h\nu}{4\pi}B_{ij}\phi_{ij}(\nu, \vec{d}),$} & \textrm{bound-bound} \\
        n_e\Phi_{ij}(T)e^{-h\nu/k_B T}\alpha_{ij}(\nu), & \textrm{bound-free},
    \end{cases}\\
%
    \label{Eq:Vji}
    V_{ji} =&
    \begin{cases}
        \makebox[\WidestCase][l]{$\frac{h\nu}{4\pi}B_{ji}\psi_{ij}(\nu, \vec{d}),$} & \textrm{bound-bound} \\
        \alpha_{ij}(\nu), & \textrm{bound-free},
    \end{cases}
\end{align}
where $\phi$ is the line absorption profile, $\psi$ is the line emission profile, $\alpha_{ij}$ is the photoionisation cross-section, $n_e$ is the local electron density, and $k_B$ is the Boltzmann constant.
$\Phi$ is the Saha-Boltzmann equation given by
\begin{equation}
    n_e\Phi_{ij}(T) = \frac{n^*_i}{n^*_j} = \frac{g_i}{2g_j}\left( \frac{h^2}
    {2\pi m_e k_B T} \right)^{3/2} \exp{\left(  \frac{\Delta E_{ji}}{k_B T}\right)},
\end{equation}
where $n^*$ is the population of the species in LTE, $m_e$ is the electron mass, $\Delta E_{ji}$ is the energy difference between levels $j$ and $i$, and $g_i$ is the statistical weight of level $i$.
By convention we define $U_{ij} = U_{ii} = V_{ii} = 0$ and $\chi_{ij} = -\chi_{ji}$.
From these definitions we see that the $U$ quantities relate to spontaneous emission, whereas the $V$ quantities describe stimulated processes.
In the case of complete redistribution, where $\psi = \phi$, all of the $U$ and $V$ terms are constant at each location in a given atmosphere, and the variation in emissivity and opacity depends on the atomic populations.

\subsection{Radiative Rates}

The radiative rates that are needed to solve the kinetic or statistical equilibrium equations are an expression of the number of upward ($ij$) and downward ($ji$) transitions due to absorption, spontaneous and stimulated emission processes (and the equivalent bound-free processes).
The formulation of emissivity and opacity through $U$ and $V$ also allows us to express the radiative rates succinctly for both upwards and downwards transitions in lines and continua at each location in the atmosphere as
\begin{equation}
    \label{Eq:RadiativeRates}
    R_{ll^\prime} = \oint \int \frac{1}{h\nu} \left( U_{ll^\prime}(\nu, \vec{d}) + V_{ll^\prime}(\nu, \vec{d})I(\nu, \vec{d}) \right)\,d\nu\,d\Omega,
\end{equation}
where $I(\nu, \vec{d})$ is the specific intensity at this location for a given frequency $\nu$ and direction $\vec{d}$.
It is through $I$ that the non-locality of the radiation field enters the problem.

\subsection{General Source Function}

Where multiple atomic species are present, the total emissivity and opacity are simply the sum of the emissivity and opacity for every transition on each atom at the current frequency and direction. It is common to additionally consider scattering by processes such as Thomson scattering, in which case the source function will be written
\begin{equation}
    S(\nu, \vec{d}) = \frac{\eta_\mathrm{tot}(\nu, \vec{d}) + \sigma(\nu)J(\nu)}{\chi_\mathrm{tot}(\nu, \vec{d})},
\end{equation}
where the ``tot'' subscript refers to these terms being summed over all interacting species, $\sigma$ describes the frequency-dependent coherent and isotropic continuum scattering cross-section, and
\begin{equation}
    J(\nu) = \frac{1}{4\pi}\oint I(\nu, \vec{d})\, d\Omega
\end{equation}
% NOTE(cmo): This angle-average term is what is computed directly from the sum_updown sum_i 0.5 * wmu_i * I, i.e., the averaging is rolled in when we use the integration. The 1/4pi vanishes due to normalisation and the fact we're in ``mu-space''. The 4pi terms all come back in wlambda, but check setup_wavelength on the _atom_ for this, it's not super clear.
is the angle-averaged intensity at frequency $\nu$.

\subsection{Iterative Solutions}

Now we have an expression for the radiative rates in each transition that can be computed numerically given the local value of the intensity, however the radiation field is not known \textit{a priori}.
Clearly, an iterative scheme will be needed to find a stable set of populations yielding a self-consistent radiation field.

If we treat the formal solver as an operator $\Lambda$ yielding the intensity from the source function throughout the atmosphere, i.e.
\begin{equation}
    I(\nu, \vec{d}) = \Lambda_{\nu,\vec{d}}[S(\nu, \vec{d})],
    \label{Eq:LambdaOperator}
\end{equation}
then starting from an initial estimate of the atomic populations (e.g. LTE) we can compute the radiation field throughout the atmosphere and iteratively use this to update the populations by solving \eqref{Eq:KinEq}. This is known as Lambda iteration (after the operator used) and presents woefully poor convergence in optically thick conditions as the size of the population updates stagnate long before the true NLTE populations are obtained.
Each additional Lambda iteration performed effectively accounts for photons that were scattered an additional time (i.e. the first Lambda iteration accounts for photons unscattered after emission, the second for once-scattered\ldots).
In the cores of optically thick lines photons will scatter a vast number of times and thus an equivalent number of Lambda iterations will be required.

The failure of Lambda iteration can be remedied by a process known as operator splitting, first introduced by \citet{Cannon1973} whereby we set
\begin{equation}
    \Lambda = \Lambda^* + (\Lambda - \Lambda^*),
\end{equation}
with $\Lambda^*$ an approximation of $\Lambda$. The iterative scheme them becomes
\begin{equation}
    I(\nu, \vec{d}) = \Lambda_{\nu, \vec{d}}^*[S(\nu, \vec{d})] + (\Lambda_{\nu, \vec{d}} - \Lambda_{\nu, \vec{d}}^*)[S^{\dagger}(\nu, \vec{d})],
    \label{Eq:Ali}
\end{equation}
where $\dagger$ identifies values from the previous iteration. This method is termed accelerated Lambda iteration (ALI), and can be shown to accelerate convergence by significantly amplifying the size of the corrections that would be computed by Lambda iteration at large optical depths, for an appropriately chosen $\Lambda^*$.
From \eqref{Eq:Ali} we can see that it is necessary to invert $\Lambda^*$ to obtain the updated value of $S$, on which $I$ is also dependent.
For a two-level atom this is discussed at length in Chaps. 12 and 13 of \citet{Hubeny2014}.
The full coupling of the terms in the multilevel NLTE problem will be made explicit in Sec.~\ref{Sec:MALI} when the MALI methods are presented; for now it is clear that the source function depends on the emissivity and opacity of each species, which in turn are controlled by the atomic populations, which are affected by the local radiation field.

A good choice for $\Lambda^*$ is not immediately evident, as it should be cheap to construct and invert, whilst providing a good approximation of $\Lambda$.
\citet{Scharmer1981} presented an upper triangular approximate operator that fits these criteria and showed its relation to the core-saturation approach of \citet{Rybicki1972} (where the net rates in the line core and wing are treated separately to precondition the net radiative rates by removing the large proportion of photons that are emitted in the wing and immediately reabsorbed).

\citet{Olson1986} proposed the use of the diagonal of the true $\Lambda$ operator as an approximate operator, and showed that this is close to optimal, and is clearly trivial to invert (as it is a scalar at each location in the atmosphere).
Now, the diagonal of $\Lambda$ is easy to obtain by setting a test source function $\mathcal{S}=\delta_{dd^\prime}$ (where $\delta$ is the Kronecker delta) and computing
\begin{equation}
    \Lambda^* = \Lambda[\mathcal{S}].
\end{equation}
Taking the example of the linear short characteristic formal solver presented in Sec.~\ref{Sec:ShortChar} and substituting this definition of $\mathcal{S}$ we obtain
\begin{equation}
    \Lambda^*_{\nu, \vec{d}} = \frac{\Delta - 1 + \exp(-\Delta)}{\Delta},
\end{equation}
where $\Delta$ is defined as in Sec.~\ref{Sec:ShortChar}.
The approximate operator can be computed analogously for other formal solvers.

\subsection{Solving the Multilevel NLTE Problem}\label{Sec:MALI}

Starting from the radiative transfer equation \eqref{Eq:1DRte} and the kinetic equilibrium equation \eqref{Eq:KinEq} we can construct a framework with which to solve the multilevel NLTE problem.
The primary term of interest is the right-hand side of \eqref{Eq:KinEq}, which is concerned with the atomic transition rates.
This is also shared with the statistical equilibrium equations \eqref{Eq:StatEq}, and thus we will solve the latter of these and return to the former later.
We follow the approach of \citet{Rybicki1992} and \citet{Uitenbroek2001}, which is known as Multi-level Accelerated Lambda Iteration (MALI) with full-preconditioning.
The full-preconditioning approach handles arbitrary overlaps of lines and continua for multiple multilevel atoms.
Earlier MALI methods \citep{Rybicki1991} are capable of handling lines overlying a constant background continuum, but not interacting with each other.
These methods describe a form of radiative transfer that is still based on a transition by transition approach to determining the radiative rates (which does not preclude self-consistency).
\citet{SocasNavarro1997} proved that this approach can be numerically equivalent to the linearised ALI method of \citet{Scharmer1985} implemented in the MULTI code \citep{Carlsson1986,Carlsson1992} provided the same assumptions are made in the design of the local operator.
Note that we distinguish between the linearised ALI method and the older method of complete linearisation \citep[e.g. ][]{Auer1969, Auer1973, Auer1976}.
The latter of these is significantly more computationally demanding and less numerically stable than the ALI based methods we are considering here \citep[a comparison of MULTI and LINEAR-B is shown in][]{Carlsson1986}.

As MALI with full-preconditioning directly supports all forms of overlapping transitions, and the interactions between them, whilst retaining the convergence properties of the ALI method, we focus solely on this method in the following description of radiative transfer.
Whilst there are also more rapidly converging methods (which can be viewed as extensions and variants of the ALI approach) such as the Gauss-Seidel and successive-over-relaxation methods of \citet{TrujilloBueno1995} (with the extension to multilevel systems presented by \citet{Paletou2007}), the forth-and-back implicit Lambda iteration of \citet{AtanackovicVukmanovic1997} (extended to multilevel systems by \citet{Kuzmanovska2017}), and even the multi-grid method of \citet{FabianiBendicho1997}, we opt for this well-tested method that is proven stable and reliable in a wide variety of situations.
This is particularly key when considering modelling of flares, which present much larger variations in parameters and steeper gradients than the quiet Sun models or academic semi-empirical models often considered.
The other, more rapidly convergent, methods present possible avenues of improvement for \Lw{} in the future.

Now, following \citet{Rybicki1992} and \citet{Uitenbroek2001} and substituting \eqref{Eq:LambdaOperator} into \eqref{Eq:StatEq}, and subsequently expanding the radiative rates gives
\begin{equation}
\begin{aligned}
   \sum_{l^\prime\neq l} &(n_{l^\prime}C_{l^\prime l}) +
   \sum_{l^\prime\neq l} \oint \int \frac{1}{h\nu} n_{l^\prime} (U^\dagger_{l^\prime l} + V^\dagger_{l^\prime l} I(\nu, \vec{d}))\, d\nu\, d\Omega\\
   -
   n_l &\sum_{l^\prime\neq l} C_{l l^\prime} -
   n_l \sum_{l^\prime\neq l} \oint \int \frac{1}{h\nu} (U^\dagger_{l l^\prime} + V^\dagger_{l l^\prime} I(\nu, \vec{d}))\, d\nu\, d\Omega
   = 0.
   \label{Eq:StatEqExpanded}
\end{aligned}
\end{equation}
$U$ and $V$ are marked with daggers so as to later incorporate the necessary PRD effects.
\citet{Rybicki1992} defined a new operator $\Psi$ such that
\begin{equation}
    \Psi_{\nu, \vec{d}}[\eta] = \Lambda_{\nu, \vec{d}}[(\chi^\dagger)^{-1}\eta]
\end{equation}
These two operators, $\Lambda$ and $\Psi$ are equivalent for a converged solution as $\chi^\dagger = \chi$, but the use of $\Psi$ is necessary to obtain the form of statistical equilibrium equations preconditioned to be linear in the atomic populations that is core to this method.

The operator splitting technique is again applied here. We then have
\begin{equation}
    I(\nu, \vec{d}) = \Psi_{\nu,\vec{d}}^*[\eta(\nu, \vec{d})] + (\Psi_{\nu,\vec{d}}- \Psi_{\nu, \vec{d}}^*)[\eta^\dagger(\nu, \vec{d})],
\end{equation}
and then considering the effects on one atom, under the assumption that background emissivity and opacity do not change during an iteration
\begin{equation}
    I(\nu, \vec{d}) = I^\dagger(\nu, \vec{d})
                    - \sum_j\sum_{i<j}\Psi_{\nu,\vec{d}}^*[n^\dagger_j U^\dagger_{ji}]
                    + \sum_j\sum_{i<j}\Psi_{\nu,\vec{d}}^*[n_j U^\dagger_{ji}],
\end{equation}
where $i$ and $j$ refer to the levels present in the atomic model.
The first two terms of this expression are often termed $I^\mathrm{eff}$ and in the case of a diagonal $\Psi^*$ operator this represents the non-local contribution to the radiation field from the current atom, and the contribution from all other species.

We can write the preconditioned statistical equilibrium system \eqref{Eq:StatEqExpanded} as
\begin{equation}
    \Gamma \vec{n} = \vec{0},
    \label{Eq:StatEqMatVec}
\end{equation}
where $\Gamma$ is a matrix consisting of the sum of $\Gamma^R$ due to the radiative contributions, and $\Gamma^C$ from the collisional contributions. $\vec{n}$ is the vector of the updated level populations for the species at this point in the atmosphere.
We can then write
\begin{equation}
    \Gamma^R_{ll^\prime} = \bigointss \bigintss \frac{1}{h\nu} \left( U^\dagger_{l^\prime l} + V^\dagger_{l^\prime l}I_{\nu, \vec{d}}^\mathrm{eff} - \left(\sum_{m\neq l}\chi^\dagger_{lm}\right) \Psi^*_{\nu, \vec{d}} \left[ \sum_p U^\dagger_{l^\prime p} \right] \right)\, d\nu\,d\Omega
    \label{Eq:GammaR}
\end{equation}
for $l\neq l^\prime$.
The problem is now represented by a system of equations linear in the level populations.
Due to the necessity of total number conservation the sum of the entries in each column of $\Gamma$ must be 0, which allows us to compute the diagonal terms as
\begin{equation}
    \Gamma_{ll} = -\sum_{m\neq l} \Gamma_{ml}.
\end{equation}

An additional constraint must be applied to \eqref{Eq:StatEqMatVec} to determine the new populations, and avoid the trivial solution of $\vec{n} = \vec{0}$, which is otherwise always a valid solution.
This constraint is typically expressed through local population conservation, which amounts to replacing one of the rows of $\Gamma$ with ones, and the associated entry in the right-hand-side zero-vector with the local species number density $n_\mathrm{total}$.
There is then a $\Gamma$ matrix at each point in the atmosphere for each atomic species being considered in detail; each of these is square with dimension equal to the number of levels in the atomic model for this species.
This system can then be solved for $\vec{n}$, independently at each location in the atmosphere, as the current estimate of non-local coupling through the radiation field is already present in the terms that compose $\Gamma$.

Iterating the populations through \eqref{Eq:StatEqMatVec} with interleaved formal solutions gives us a reliable and rapidly converging method for solving the multilevel NLTE problem with multiple atoms and overlapping transitions as used in the \Lw{} framework.
A more detailed description of the numerical implementation is provided in the \Lw{} paper \citep{Osborne2021}.

\subsection{Time-Dependent Population Updates}\label{Sec:TimeDepPopUpdates}

As discussed previously, an iterative update to the populations in the statistical equilibrium case can be phrased as $\Gamma \vec{n} = \vec{0}$.
A variant of this method can also be applied to the time-dependent form of the kinetic equilibrium equations \eqref{Eq:KinEq}.
We will not directly treat the advection term of equation \eqref{Eq:KinEq}, as this requires consideration of hydrodynamic effects and the discretisation schemes used therein (due to the large gradients of these populations that occur in the solar atmosphere)\footnote{An overview of the methods used to numerically solve the conservation laws of hydrodynamics is provided in Sec.~\ref{Sec:IntroConsLaws}.}.
We can, however, discretise $\partial n / \partial t = \Gamma \vec{n}$ as
\begin{equation}
    \label{Eq:ThetaDisc}
    \frac{\vec{n}^{t+1} - \vec{n}^t}{\Delta t} = \theta \Gamma^{t+1} \vec{n}^{t+1} + (1-\theta)\Gamma^{t} \vec{n}^{t},
\end{equation}
where $\theta$ indicates the degree of implicitness, the $t$ and $t+1$ indices indicate the start and end of the timestep being integrated respectively, and $\Delta t$ the duration of the timestep.
$\vec{n}^{t+1}$ can then be found by rewriting \eqref{Eq:ThetaDisc} as
\begin{equation}
    \label{Eq:TimeDepSystem}
    (\mathbb{I} - \theta\Delta t \Gamma^{t+1}) \vec{n}^{t+1} = (1-\theta)\Delta t \Gamma^{t}\vec{n}^{t} + \vec{n}^{t},
\end{equation}
with $\mathbb{I}$ the identity matrix.
This system is then iterated until $\vec{n}^{t+1}$ converges, for a new evaluation of $\Gamma^{t+1}$ at each iterate.

\subsection{Partial Frequency Redistribution}\label{Sec:Prd}

Most solar spectral lines form in regions where complete frequency redistribution (CRD) holds.
That is to say that the plasma is sufficiently collisional that elastic collisions redistribute electrons across all sub-states of an energy level prior to emission.
In this case, the emission frequency of a photon is not correlated with the frequency of the photon absorbed to excite the atom into this state i.e. photons are completely redistributed in frequency and the line emission and absorption profiles are equal.
In lower density regions with strong, typically resonance\footnote{A transition whose lower state is the ground state of the atom is known as a resonance transition.}, lines where radiative effects dominate over collisional effects, there is said to be a natural population of a certain level: a population where the emission frequency is correlated to the absorption frequency, and has not been redistributed by collisions \citep{Hubeny2014}.
The line's emission and absorption profiles then differ and this coherent scattering must be treated explicitly.
This imposes substantial computational effort, but we will briefly describe the key points of the process, and how it fits into the MALI framework following \citet{Uitenbroek2001} and \citet{Hubeny2014}.

We can define the emission profile coefficient $\rho_{ij} = \psi_{ij} / \phi_{ij}$, at which point all $U$ and $V$ terms can be rewritten in terms of $\rho_{ij}$ and $\phi$, and $\dagger$ on these terms refers to the value of $\rho_{ij}$ evaluated at the previous iterate (analogous to $n^\dagger$).

From this definition of $\rho_{ij}$, under the assumptions of a line with an infinitely sharp lower level and broadened upper level, and the validity of PRD in the atomic frame being approximated by PRD in the observer's frame \citep{Uitenbroek2001}, following \citet{Hubeny2014} we have
\begin{align}
\begin{split}
    \rho_{ij}(\nu, \vec{d}) = 1 + &\gamma\frac{\sum_{l < j}n_j B_{lj}}{n_j P_j} \oint\frac{1}{4\pi}\int I(\nu^\prime, \vec{d}^\prime) \\ &\cdot \left[ \frac{R^{II}_{lji}(\nu^\prime, \vec{d}^\prime; \nu, \vec{d})}{\phi_{ij}(\nu, \vec{d})} - \phi_{lj}(\nu^\prime, \vec{d}) \right]\,d\nu^\prime\,d\Omega^\prime,
\end{split}
\end{align}
where $R^{II}$ is the generalised redistribution function for transitions of this kind \citep{Hubeny1982}, and $\gamma$ is the coherency fraction.
The $lji$ subscript on $R^{II}$ describes the scattering process, indicating that electrons can start from any level $l$ in the range $[i, j)$.
These scattering processes are then summed.
The cross-redistribution, or Raman scattering, processes for which $l\neq i$ are typically less important than resonance scattering ($l = i$).
Only considering the latter of these simplifies the evaluation of $\rho_{ij}$.
The redistribution function describes the probability that a photon with frequency $\nu^\prime$ and direction $\vec{d}^\prime$ is re-emitted with frequency $\nu$ and direction $\vec{d}$.
% The integration over frequency and solid angle involving the product of the intensity and the redistribution function is often termed the \emph{scattering integral}, and requires careful numerical integration to ensure correct normalisation.

The coherency fraction $\gamma$ describes the probability of a photon being emitted from its current sublevel of the energy level $j$, prior to an elastic collision that would redistribute it across the sublevels of $j$.
This is computed as
\begin{equation}
    \gamma = \frac{P_j}{P_j + Q_j},
\end{equation}
where $P_j$ is the total depopulation rate of level $j$, and $Q_j$ is the total rate of elastic collisions affecting this level.

Ignoring bulk plasma motions that create anisotropy in the radiation field, the integrals over angle and frequency can be split, and render the calculation of this angle-averaged form of $\rho_{ij}$ much simpler and less computationally demanding.
Also defining $g_{II}(\nu, \nu^\prime) = R^{II}(\nu, \nu^\prime)/\phi_{ij}(\nu^\prime)$ such that the fast approximation of \citet{Gouttebroze1986} and \citet{Uitenbroek1989} can be employed, and ignoring cross redistribution terms we have
\begin{equation}
    \label{Eq:RhoPrdAa}
    \rho_{ij}(\nu) = 1 + \gamma \frac{n_i B_{ij}}{n_j P_j} \left( \int g_{II}(\nu, \nu^\prime) J(\nu^\prime)\,d\nu^\prime - \bar{J}_{ij} \right),
\end{equation}
where
\begin{equation}
\bar{J}_{ij} = \frac{1}{4\pi} \oint \int I(\nu, \vec{d}) \phi(\nu, \vec{d})\,d\nu\,d\Omega = \frac{R_{ij}}{B_{ij}}
\end{equation}
is the frequency-integrated mean intensity across the transition.
In the case of plasma flows we adopt the approximate hybrid treatment of \citet{Leenaarts2012}, using $J$ in the plasma's rest frame, to compute $\rho_{ij}$ in this same frame.
Its value is then interpolated to find the value of $\rho_{ij}$ at the Doppler shifted value in the observer's frame.
The additional computational cost of this hybrid method over the angle-averaged method is very low, and in most cases the results are comparable to the far more costly complete angle-dependent treatment \citep{Leenaarts2012, Kerr2019}.

We adopt the iterative method of \citet{Uitenbroek2001} to evaluate $\rho_{ij}$ by interleaving formal solutions at wavelengths of PRD lines (and wavelengths that are Doppler shifted into the region of a PRD line in the case of hybrid PRD) between updates of $\rho_{ij}$ from the current value of the radiation field.
The atomic populations are held constant during this process, and a handful of iterations of computing $\rho_{ij}$ and updating the radiation field are performed between each MALI iteration.


\subsection{Charge Conservation}\label{Sec:ChargeCons}

We have thus far assumed that the electron density is known \emph{a priori} as part of the atmospheric model.
Unfortunately, it is also insufficient to assume that the electron density follows the LTE ionisation state of the plasma, as many species (in particular hydrogen) with important NLTE spectral lines will be far from their LTE ionisation state \citep{Heinzel1995,Paletou1995,Bjorgen2019}.
Synthesis of lines with an incorrect electron density will often converge to different final populations, and produce different spectral line shapes.
A secondary iteration process is needed to determine the electron density in self-consistent way within the MALI framework.
This is achieved through a Newton-Raphson iteration first proposed by \citet{Heinzel1995} and \citet{Paletou1995}.
The method presented here differs slightly from that presented by these authors as we choose to include the effects of all species considered in NLTE in the calculation of charge conservation.

The statistical equilibrium equations of level $i$ of species $s$ can be written as a function of the species' level populations and the electron density
\begin{equation}
    \label{Eq:EseFn}
    F_{s, i}(\vec{n}_s, n_e) = \sum_{j\neq i} n_j P_{ji}(\vec{n}_s, n_e) - n_i\sum_{j\neq i} P_{ij}(\vec{n}_s, n_e) = 0.
\end{equation}
Under our previous assumption of fixed electron density this expression is linear in unknown populations and reduces to the preconditioned system of the MALI method \eqref{Eq:StatEqMatVec}.
We first obtain the solution to this system and denote this intermediate result $\widetilde{\vec{n}_s}$.

A Newton-Raphson iteration is applied to this system to compute the correction to the level populations and electron density that will drive the $F_{s,i}(\vec{n}_s, n_e)$ towards 0.
This is achieved by using the Jacobian of $F$.
For an arbitrary function $f(\vec{x})$, to compute the correction $\delta x$ to an initial parameter $\vec{x_0}$ such that $f(\vec{x_0} + \vec{\delta x}) = 0$ it is sufficient to solve $-J\vec{\delta x} = f(\vec{x_0})$ for $\vec{\delta x}$, where $J$ is the Jacobian of $f$ evaluated at $\vec{x_0}$.
Applying this procedure to the preconditioned statistical equilibrium equations gives
\begin{equation}
    \label{Eq:LinNr}
    F_{s, i}(\widetilde{\vec{n}_s}, n_e^\dagger) =
    - \sum_j \left( \left.\frac{\partial F_{s,i}(\vec{n}_s, n_e)}{\partial n_j}\right\rvert_{(\widetilde{\vec{n}_s}, n_e^\dagger)} \delta n_j \right)
    - \left.\frac{\partial F_{s,i}(\vec{n}_s, n_e)}{\partial n_e}\right\rvert_{(\widetilde{\vec{n}_s}, n_e^\dagger)} \delta n_e,
\end{equation}
where $\delta n_j$ and $\delta n_e$ are the necessary corrections to obtain self-consistent level populations and electron density.

This system applies \emph{simultaneously} to all species to be considered in the determination of the self-consistent electron density, and contains $\sum_s N_{\mathrm{level}, s} + 1$ equations, where $N_{\mathrm{level}, s}$ is the number of levels in the model of species $s$.
Similarly to the initial population update through the MALI method \eqref{Eq:StatEqMatVec}, constraints are needed to ensure that all parameters are correctly conserved.
Here we require constraints on the population of each species, and on the charge neutrality of the system.
The former of these is expressed as
\begin{equation}
    \sum_j \delta n_j = n_\mathrm{total} - \sum_j \widetilde{n_j},
\end{equation}
for each species, while the latter is described by
\begin{equation}
    \delta n_e - \sum_s \sum_j \mathion_s(j) \delta n_{s, j} = n_{e, \mathrm{bg}} + \sum_s \sum_j \mathion_s(j) - n_e^\dagger,
\end{equation}
where $\mathion_s(j)$ is the ionisation state of the $j$-th level of species $s$, and $n_{e, \mathrm{bg}}$ represents the electron density due to background species not considered in detail here.
The left-hand side of the population conservation equation replaces one of the rows of the Jacobian for each species, analogously to the population conservation equation used in \eqref{Eq:StatEqMatVec}.
The charge conservation equation is the ``extra'' equation in this system, and the left-hand side of the expression here forms one row of the Jacobian.

The final Jacobian matrix is therefore block diagonal, with one row that couples these blocks to each other.
The left-hand side of the system \eqref{Eq:LinNr} is given by the right-hand sides of the previous constraint equations, and $F_{s, i}(\widetilde{\vec{n}_s} n_e^\dagger) = \Gamma_s \widetilde{\vec{n}}$ for the remaining entries for each species $s$.
This system can now be solved in the same fashion as \eqref{Eq:StatEqMatVec}, and the final populations can be computed from $n_j = \widetilde{n_j} + \delta n_j$ and $n_e = n_e^\dagger + \delta n_e$.
The exact form of these derivatives used in the Jacobian can be computed analytically for all rates, and these terms are presented in \citet{Osborne2021}.
The derivatives with respect to the level populations are simply the associated entries of $\Gamma$ for each species, whilst those associated with the electron density depend on the collisional and bound-free radiative rates, but we do not find the exact form to be particularly insightful.

The time-dependent case can be solved very similarly to the previous statistical equilibrium case.
Following \citet{Kasparova2003}, and starting from the time-dependent population update equation \eqref{Eq:TimeDepSystem}, we can define
\begin{equation}
    \label{Eq:TimeDepNrFn}
    G_{s,i}(\vec{n}^{t+1}_s, n_e) = n^{t+1}_{s,i} - \theta \Delta t F_{s,i}(\vec{n}^{t+1}_s, n_e) - (1-\theta)\Delta t \Gamma^t_s \vec{n}^t_s - n^t_{s,i} = 0,
\end{equation}
and apply this secondary Newton-Raphson iteration process to $G$, using the same constraint equations as before.

It is important to stress that only one Newton-Raphson iteration is needed following each standard preconditioned population update, as the original MALI system was already preconditioned and linear in the populations.
The iterative process proceeds by following each MALI population update with a secondary Newton-Raphson iteration evaluated using the updated $\Gamma$, and then returning to the evaluation of $\Gamma$, or $\rho_{ij}$ in the presence of transitions treated with PRD.

\subsection{Determining Numerical Convergence}

The iterative procedures presented in the previous sections yield refined estimates of the level populations throughout the model atmosphere for each additional iteration.
This raises the question of when to stop iterating, and by what metric to measure convergence.
Convergence describes the difference between these values for two subsequent iterations, and how far they are from the true values for the model.
As we do not generally have knowledge of the latter, the former is typically used to determine when to cease iterations.
The most important quantities to use when estimating convergence are the populations $n$, and radiation field $J$, due to their appearance in the source function.

In NLTE problems, where spectral lines will often form in relatively compact regions of the atmosphere, the change may only be large in this region, so the $L^1$ or $L^2$ norms of the difference between successive iterates may not be very informative.
Additionally, the value of both $J$ and $n$ will vary hugely throughout the atmosphere, thus the absolute change over an iteration may not be particularly meaningful.
Instead, it is common to use the $L^\infty$ norm\footnote{The $L^\infty$ norm of a vector is given by the maximum absolute value present in its components.} of the relative change of these parameters.
For a population $n_i$, following \citet{Auer1994a}, we denote the relative change $R_c(itr, i)$ for level $i$ and iteration $itr$.
It can be expressed as
\begin{equation}
    R_c(itr, i) = \mathrm{max}_k \left|\frac{n_i^{itr} - n_i^{itr-1}}{n_i^{itr}}\right|,
\end{equation}
where $k$ represents the location in the atmosphere, and $n_i^{itr}$ represents the population of level $i$ after iteration $itr$.
By iterating until $R_c(itr, i)$ is low for all transitions, we ensure that the largest update to a term is small relative to its current value.
When applied to $J$, the $L^\infty$ norm of the relative change is typically computed across frequency and atmospheric depth.

\citet{Auer1994a} and \citet{FabianiBendicho1997} provide a framework for estimating both the error from the fully converged solution, and the truncation error due to the discretisation onto the numerical grid, but in practice this is rarely employed outside of multi-grid strategies \citep[e.g.][]{FabianiBendicho1997,Leger2007}.
Instead, we apply the more common technique of iterating until the maximum value of $R_c$ for the populations drops below a certain threshold (typically $\sim 10^{-3}$)\footnote{It is wise to also track $R_c$ for $J$, and although this is less commonly used as a convergence criterion, we often choose to do so as a safety factor. The $R_c$ of one of $n$ or $J$ being significantly larger than the other likely indicates that the spatial, frequency, or angular quadrature is ill-suited to the problem at hand.}.
This approach has been applied with success to many modelling problems within the MULTI, RADYN, and RH \citet{Uitenbroek2001} codes (amongst others), but we note, following \citet{Auer1994a} and \citet{FabianiBendicho1997}, that a small value of $R_c$ does not guarantee convergence, whilst methods such as the multi-grid one they present can do so.
Unfortunately, these methods are significantly more complex than the MALI method presented here and appear to have difficulty in more realistic multi-dimensional models (J. Štěpán, \emph{private communication}).


\section{Introduction to Hydrodynamics and Conservation Laws}\label{Sec:IntroConsLaws}
\emph{The majority of the basic theory in this section follows the two texts by \citet{LeVeque1997,LeVeque2002}.}

In the previous sections we have explained the basis of the radiative transfer methods used throughout this work.
The difficulty in radiative transfer primarily lies in the nuance of implementing the various integration terms.
To provide a clear understanding of RHD we also need a numerical description of hydrodynamics, and an explanation of solving the coupled systems of partial differential equations (PDE) that represent the other major facet of RHD.

The scalar radiative transfer equation, solved via the formal solver, is a good example of an ordinary differential equation.
It is relatively easy to solve via a variety of methods, typically striving for a balance of speed, reliability, and accuracy.
Unfortunately, PDEs are difficult to solve numerically in a general way.

A generic second-order PDE of a quantity $q$ depending on two independent variables $x$ and $y$ can be written as
\begin{equation}
    aq_{xx} + bq_{xy} + cq_{yy} + dq_x + eq_y + fq = g,
\end{equation}
wherein the subscripts represent partial derivatives with respect to these variables.
For convenience we will retain this convention through the current section.
The sign of the discriminant ($\Delta = b^2-4ac$) of this equation determines the class of the problem:
\begin{itemize}
    \item $\Delta < 0$: Elliptic problem (e.g. Poisson equation).
    \item $\Delta = 0$: Parabolic problem (e.g. Heat equation).
    \item $\Delta > 0$: Hyperbolic problem (e.g. Advection equation).
\end{itemize}
Here, we will focus primarily on hyperbolic problems, with a brief discussion of parabolic terms.
Hyperbolic problems take the form
\begin{equation}\label{Eq:ConsLaw}
    q_t + f(q)_x = 0,
\end{equation}
where $f$ is a function describing the the flux of $q$ at each location in the domain, $t$ indicates a temporal coordinate, and $x$ a spatial coordinate.
In the following discussions, there is an inherent assumption that the flux function is local, and acts only on the local state variables $q$.
This is not necessarily the case with radiative terms, but these can be incorporated into such a scheme by determining their local effects on the plasma energy.

Equations of the form \eqref{Eq:ConsLaw} are known as conservation laws, and the quantities $q$ are often termed ``conserved quantities'', implying that $\int_{-\infty}^{\infty} q\, dx$ is constant in time.
This does not preclude the addition of sources and sinks of this quantity, but simply requires that the total of a conserved quantity not change \emph{without} being acted on in such a way.
The simplest equation of this form is advection, which arises from conservation of mass in a moving fluid and is written for mass density $\rho$ and fluid velocity $\varv$ as
\begin{equation}\label{Eq:Advection}
    \rho_t + (\rho \varv)_x = 0.
\end{equation}

The advection equation can be augmented with two further equations to form the Euler equation set.
These three equations are the conservation of mass, momentum, and energy.
Together they describe the evolution of ideal fluids.
The complete set is written
\begin{equation}\label{Eq:EulerEqns}
\begin{split}
    \rho_t + (\rho \varv)_x &= 0,\\
    (\rho \varv)_t + (\rho \varv^2 + p)_x &= 0,\\
    E_t + (\varv(E + p))_x &= 0,
\end{split}
\end{equation}
where $E$ is the total energy and $p$ is the gas pressure.
This system is already recognisable as the simplified roots of the RHD equations.
As our three conserved quantities are the mass density, momentum density, and energy, the pressure must be expressed as a combination of these so as to be able to write this system in the form of \eqref{Eq:ConsLaw}.
The simplest way to achieve this is to use the equation of state for an ideal gas
\begin{equation}
    e = \frac{p}{\gamma-1},
\end{equation}
with $e$ the internal energy, and $\gamma$ is the ratio of gas' specific heat at constant pressure and constant volume, which is 5/3 for monatomic gases, typically assumed in the case of solar plasma.
This is related to the total energy $E$ by
\begin{equation}
    E = e + \frac{1}{2}\rho \varv^2.
\end{equation}

The Euler equations \eqref{Eq:EulerEqns} describe the evolution of an ideal fluid without any energy losses.
In practice we often need to model some loss terms.
Ignoring radiative effects for now, the most important of these is heat conduction\footnote{In the plasmas considered here there is additional nuance to heat conduction, which we will return to in Sec.~\ref{Sec:NumericalConduction}.}, which is typically described by parabolic equations.
This adds a second spatial derivative term to the right-hand side of the energy conservation equation.
It is also common to need source and sink terms when modelling real world problems; these can account for fluids entering and leaving a volume, the non-local emission and absorption of energy, or simply the effects of gravity.
These terms are also added to the right-hand side of the equations of \eqref{Eq:EulerEqns}, and together with the effects of viscosity, these describe the Navier-Stokes equations.

\subsection{Numerical Approaches}

The typical first choice for numerically solving differential equations is to apply a finite difference method.
In this class of method, the problem is discretised, similarly to the approach taken in radiative transfer, and the values associated with each grid point represent the local pointwise value.
The local gradient of the conserved quantities can then be estimated from pointwise difference between in the quantity between adjacent cells.
This method can be applied to discrete problems in both space and time.
Applying a basic one-sided method to the advection equation \eqref{Eq:Advection} gives
\begin{equation}\label{Eq:FdmAdvection}
    \frac{q^{t+1}_i - q^t_i}{\Delta t} + \varv \left( \frac{q^t_{i+1} - q^t_i}{\Delta x} \right) = 0,
\end{equation}
where the subscript refers to the location of the conserved quantity, the superscript the discrete timestep, with $\Delta x$ and $\Delta t$ being the local grid spacing and timestep duration respectively.

It is clear that without loss of generality we could have also chosen to spatially difference our problem in the other direction (i.e. $q^t_i - q^t_{i-1}$).
From the infinitesimal definition of the derivative, these two formulations are equivalent.
This is not the case in discretised problems.
It is better to locally use the formulation such that information from points ``upwind'' in terms of the fluid velocity are used to update the points downwind of themselves.
In this sense the information used to update points is following the fluid flow.
Equation \eqref{Eq:FdmAdvection} can be rewritten to explicitly determine the approximate value of the quantity at the next timestep,
\begin{equation}
    q^{t+1}_i = q^t_i - \frac{\varv \Delta t}{\Delta x}\left( q^t_{i+1} - q^t_i \right).
\end{equation}

This simple first order accurate approach can be applied to any conservation law, and higher-order accurate methods can be derived from using finite difference methods over larger stencils, or deriving similar approaches from combinations of the local finite difference approximations using Taylor series.

Many other spatial and temporal discretisations can be devised for conservation laws, and it is important to choose a discretisation that introduces minimal error.
In general, we term discretisations where $q^{t+1}$ depends only on $q^{t}$ as \emph{explicit}, and those also depending on $q^{t+1}$ as \emph{implicit}, necessitating the solution of a system of typically non-linear equations.

Whilst the finite difference method provides an intuitive formulation for discretising these equations, it is difficult (but entirely possible) to ensure conservation of the quantities that we desire be conserved with only pointwise values and no formal description of the variation between these points.
Instead, the finite volume description is often preferable and consists of treating $q_i^t$ as the average value over grid cell $i$.
Conservation can then be ensured by evolving this value based on the fluxes in and out of the cell.
This implies
\begin{equation}
    q_i^t \approx \frac{1}{\Delta x}\int_{x_i}^{x_{i+1}} q^t(x)\, dx.
\end{equation}
Rewriting the conservation law \eqref{Eq:ConsLaw} in an integral form then gives
\begin{equation}
    \int_{x_i}^{x_{i+1}} q^{t+1}(x)\, dx = \int_{x_i}^{x_{i+1}} q^{t}(x)\, dx
                                         + \int_{t_t}^{t_{t+1}} q_i(t)\, dt
                                         - \int_{t_t}^{t_{t+1}} q_{i+1}(t)\, dt,
\end{equation}
which can then be written as the flux-differencing form
\begin{equation}\label{Eq:FiniteVolumeMethod}
    q_i^{t+1} = q_i^t - \frac{\Delta t}{\Delta x}\left( F^t_{i+1} - F^t_{i} \right),
\end{equation}
where $F_i$ represents the flux due between cells $i$ and $i-1$, and $F_{i+1}$ the flux between cells $i$ and $i+1$.
In the case of an explicit method with a correctly chosen timestep for the grid, where information cannot move further than one cell in a timestep, each of these flux functions depends only on cell $i$ and one of its neighbours.
An important strength of this method for solving conservation law problems is that if $F_i$ and $F_{i+1}$ are respectively the left and right edge fluxes for cell $i$, then $F_{i+1}$ will be the left edge flux for cell $i+1$, and $F_{i}$ will be the right edge flux for cell $i-1$.
Thus the numerical integration of $q^{t+1}$ is conserved with respect to $q^t$, as all numerical fluxes, other than the left- and right-most, will cancel due to the formulation of \eqref{Eq:FiniteVolumeMethod}.
These left- and right-most fluxes will need to consider the boundary conditions of the finite simulation volume to ensure the correctness of the conservation law.
Equivalent formulations can be found for finite-difference approaches, but are slightly harder to arrive at.


\subsection{Riemann Problems}

% spell-checker: disable
% \begin{pycode}[FlareModelling]
% from shocktubecalc import sod
% from matplotlib.ticker import MaxNLocator

% gamma = 1.4
% positions, regions, values = sod.solve(left_state=(1.0, 1.0, 0.0), right_state=(0.1, 0.125, 0.0),
%                                        geometry=(0.0, 1.0, 0.5), t=0.2, gamma=gamma, npts=500)
% fig, ax = plt.subplots(2, 2, figsize=texfigure.figsize(pytex, scale=1, height_ratio=0.6),
%                        constrained_layout=True)
% ax = ax.ravel()
% grid = values['x']
% pressure = values['p']
% density = values['rho']
% velocity = values['u']
% totE = pressure / (gamma - 1.0) + 0.5 * density * velocity**2

% ax[0].plot(grid, density)
% ax[0].set_xticklabels([])
% ax[1].plot(grid, velocity)
% ax[1].set_xticklabels([])
% ax[2].plot(grid, pressure)
% ax[3].plot(grid, totE)

% ax[0].set_ylabel(r'$\rho$')
% ax[1].set_ylabel(r'$\varv$')
% ax[2].set_ylabel(r'$p$')
% ax[3].set_ylabel(r'$E$')

% ax[2].set_xlabel(r'$x$')
% ax[3].set_xlabel(r'$x$')

% lFig = chFlareModelling.save_figure('SodTubeAnalytic', fig, fext='.pgf')
% lFig.caption = r'Solution to the classic Sod shock tube problem at $t=\SI{0.2}{\second}$.'

% fig = plt.figure(figsize=texfigure.figsize(pytex, scale=1, height_ratio=0.55))
% ax = plt.gca()
% ax.plot(grid, density)
% ax.set_ylabel(r'$\rho$', c='C0')
% ax.set_xlabel(r'$x$')

% ax2 = ax.twinx()
% # ax2.plot([0.5, 0.7], [0, 0.2], c='#222222')
% for i, p in enumerate(positions.values()):
%     ax2.axvline(p, c='k')
%     if i != 1:
%         ax2.plot([0.5, p], [0, 0.2], c='#555555')
% regionLims = np.array([0, *positions.values(), 1])
% regionCentres = 0.5 * (regionLims[1:] + regionLims[:-1])
% labels = [r'$\alpha$', r'$\beta$', r'$\gamma$', r'$\delta$', r'$\epsilon$']
% for i, c in enumerate(regionCentres):
%     ax2.text(c, 0.02, labels[i])

% ax2.set_ylabel(r'$t$')
% ax2.yaxis.set_major_locator(MaxNLocator(5))
% lFig = chFlareModelling.save_figure('RiemannFan', fig, fext='.pgf')
% lFig.caption = r'Riemann fan of different wave speeds and associated regions for the Sod shock tube problem.'
% lFig.short_caption = r'Riemann fan of different wave speeds for the Sod shock tube problem.'
% \end{pycode}

% \py[FlareModelling]|chFlareModelling.get_figure('SodTubeAnalytic')|
% \py[FlareModelling]|chFlareModelling.get_figure('RiemannFan')|
% spell-checker: enable

The finite volume method we have described therefore depends on finding expressions for the numerical flux between two adjacent cells.
This is often framed as a Riemann problem at the cell interface.
A Riemann problem consists of solving a conservation law with an initial parameter distribution consisting of two constant states meeting at a discontinuity.
% Without loss of generality this discontinuity can be placed at $x=0$, and the value of $q$ for $x<0$ is termed $q_l$ and $q_r$ for $x>0$.

% A very well studied Riemann problem, that is often used as a test case for numerical hydrodynamics, is that of the Sod shock tube \citep{Sod1978}.
% The problem is a single Riemann problem, with gas at a different pressure and density on each side of a membrane.
% At $t=0$ the membrane is removed, and multiple waves with different velocities are launched in both directions.
% A solution to the classic Sod shock tube with initial states $q_l = \{\rho=1, p=1, \varv=0\}, q_r = \{\rho=0.125, p=0.1, \varv=0\}$ at $t=\SI{0.2}{\second}$ is shown in Fig.~\ref{Fig:SodTubeAnalytic}.
% At this time, a complex pressure and density structure has formed in the shock tube.
% The Riemann problem is often considered in terms of the propagating waves.
% In Fig.~\ref{Fig:RiemannFan} we show the same density profile, but divided into regions by black lines, with grey lines showing a time-distance plot of the different waves in the system.
% This plot, and manner of considering the waves in the system is known as a Riemann fan.

% The left- and right-most regions of density ($\alpha$ and $\epsilon$) shown in Fig.~\ref{Fig:RiemannFan}, are simply the initial left and right states of the fluid, as these are currently unperturbed by the waves propagating from the initial discontinuity.
% Starting from the initial discontinuity, a rarefaction wave (reduction in density) moves into the high density region (to the left), and a shock wave propagates into the low density region (to the right).
% The rarefaction wave is present in region $\beta$, with the shock front being the boundary between $\delta$ and $\epsilon$.
% There is an additional front between regions $\beta$ and $\gamma$, which we note is not visible on either the velocity or pressure panels of Fig.~\ref{Fig:SodTubeAnalytic}.
% This is a contact discontinuity and represents the boundary between two regions of different entropy.
% This is the original interface between the two gases at $t=0$, and it is advected along with the flow at the velocity $u$, so there is no mixing between the two sides of this contact discontinuity (in the case of an ideal gas).

This problem can be analytically solved by considering the propagation of waves from this interface.
From analysis of the Jacobian of the flux function of the Euler equations, the structure of this solution can be revealed.
There are three waves, associated with the eigenvalues of this Jacobian: $\varv$ and $\varv\pm c_s$, where $c_s$ is the sound speed of the fluid.
These waves can be non-linear and do not necessarily propagate at the characteristic velocity given by their associated eigenvalue.
The expected solution is a contact discontinuity propagating with the fluid velocity $\varv$, and two non-linear waves, a shock wave and a rarefaction wave.
That states on either side of these waves will need to be computed using an iterative process after the application of the Rankine-Hugionot jump conditions.

As the complexity of the system and the equation of state increases, it becomes harder (or impossible) to compute this solution in an efficient manner.
Fortunately, it is rarely necessary to compute the exact solution to the Riemann problem as many efficient approximate solvers exist, but the structure originating from this simple case can be used to interpret and validate other numerical methods.
This solution will need to be computed at every interface to determine the fluxes the cells, but the question remains of how to define the states left and right of the interface in the definition of the Riemann problem.

% N.B. The characteristic of a system like this is a line for which the the solution q is constant. i.e. for advection this is x(t) = x0 + ut
% For linear systems we can transform to characteristic variables, for which the coefficient matrix is diagonal and we have a system of independent scalar advection equations.
% The velocity is time/space varying along the rarefaction wave.
% The characteristic field of an equation is an eigenvector, if this is linearly
% degenerate then we can only have contact discontinuities.
% Integral curves are curves which connect two points by an integral of one of the eigenvectors of the system.
% A function of q that is invariant along any integral curve is called a Riemann invariant, these are used for solving problems related to rarefaction waves.
% These Riemann invariants are conserved along the characteristic. For the u eigenvector, we have u and p as invariant, hence only the pressure can change => contact discontinuity.
% Characteristics on each side of shock linked by Hugoniot locus

\subsection{Godunov's Method and Higher Order Reconstructions}\label{Sec:HydroReconstruction}

The method of \citet{Godunov1959} consists of assuming that the data $q$ is piecewise constant in each cell of our simulation domain.
At this point the values on the left- and right-hand side of each interface are known and the flux through this interface can be computed.
The Riemann problem at each interface can be treated independently under the assumption that the fastest wave from one interface not carry information to the next.
This is a fundamental requirement of stability and will be discussed in more detail in Sec.~\ref{Sec:HydroStability}.

This method provides a basic framework for solving conservation laws.
It is limited by the assumption that the data is piecewise constant, but nevertheless paved the way for some of the most accurate numerical methods for conservation laws.
\citet{VanLeer1979} provided one of the first higher order extension to Godunov's method.
It is tempting to attempt to estimate the value of $q$ at the cell interfaces with higher accuracy by using some form of reconstruction (a method closely related to interpolation), but care must be taken with this approach.
The Monotonic Upstream-centred  Scheme for Conservation Laws (MUSCL) method of \citet{VanLeer1979} uses a piecewise linear approach to reconstruction in each grid cell, but limits the gradient of the reconstruction to prevent the addition of under- or over-shoots to the data.
This is achieved through use of a slope-limiter such that a monotonic series of cell averages be preserved by ensuring that the reconstructed slope not take values beyond the average of the adjacent cells.
If the current cell is an extremum then the slope is set to 0.
% Different slope-limiters have been designed and present different trade-offs in terms of accuracy in smooth regions against the risk of introducing spurious oscillations in the solution, these are discussed at length in introductory texts, such as \citet{LeVeque1997}.
This method tracks the data to second-order in regions of smooth variation, but degenerates to the first-order Godunov method at discontinuities.
The reconstruction technique used in \Radyn{} is based on this method.

There are further high-order extensions to the concept of reconstruction including the parabolic method of \citet{Colella1984} providing third-order accuracy in smooth regions, and the general weighted essentially non-oscillatory (WENO) methods.
WENO methods are a cornerstone of reconstruction\footnote{The properties that make WENO methods a good choice for reconstruction also render them applicable to interpolation. Throughout \Lw{} and the other numerical tools presented in this thesis we make use of a fourth-order WENO interpolation method described by \citet{Janett2019} for its accuracy in smoothly varying regions and reliable behaviour around sharp variations. For example, in \Lw{} it is used for the interpolation of photoionisation cross-sections onto the final wavelength grid.} in modern finite volume and finite difference codes, and as such we will describe them briefly here.

WENO methods were first proposed by \citet{Liu1994}, and formalised for arbitrary order by \citet{Jiang1996}.
These methods form a convex combination of polynomials over overlapping regions.
For example, in the case of the commonly used fifth-order WENO method of \citet{Jiang1996}, three parabolae are constructed over five adjacent points, i.e. each using three contiguous points.
We can equivalently define a fourth-order polynomial over the five points, which can also be written as a linear combination of our interpolating parabolae at each point in this region.
These linear weights are further weighted by a term known as the \emph{smoothness indicator}, which estimates the local smoothness, such that the weight of each term is equal to its expected linear weight in smooth regions, and heavily biased towards a parabola in a smooth region if other regions present discontinuities.
WENO methods are highly performant thanks to the lack of conditional branches in this core procedure.

In the case of the finite volume method these functions can be derived to reconstruct the interface values from the cell average values, and for a fixed uniform grid, the integration weights for each interface can be computed analytically.
For non-uniform grids, the weights can be precalculated if the grid remains fixed, or computed on the fly if necessary.
There are also a number of approximate methods for handling non-uniform grids at low computational cost such as WENO-NM \citep{Huang2018}.

\subsection{Numerical Fluxes}

Whilst there are many approximate Riemann solvers that can accurately and efficiently solve the Riemann problems arising from the Euler equations including the addition of source terms, it does not appear feasible to express the evolution of the entire RHD system in this way.
Instead, in explicit methods we can use numerical approximations to the flux based on the reconstructed values at the interfaces.
For implicit methods we can instead use a Newton iterative scheme to minimise the residual of the discretised conservation law across the cell interfaces with the fluxes typically computed from the upwind value of the reconstructed parameter.

With high-order reconstructions, simple expressions for the fluxes can be used and give accurate results.
An obvious first choice would be the average flux from the reconstructed states left and right of the interface.
Unfortunately, this leads to numerical instabilities, and some artificial damping (numerical viscosity) is needed.
This leads to a flux known as the Local Lax-Friedrichs, or Rusanov Flux \citep{Rusanov1962}
\begin{equation}
    F_{\mathrm{LLF}} = \frac{1}{2}(f(q_{i+1}^L) + f(q_i^R)) - \frac{1}{2}\alpha(q_i^R - q_{i+1}^L).
\end{equation}
Here $q_i^L$ and $q_i^R$ represent the left- and right-hand reconstructed states of the $i$-th Riemann problem, $f$ represents the flux function of the conservation law, and $\alpha$ is the local maximum wave propagation speed for this system.
Whilst $\alpha$ can often be formally derived from the Jacobian of $f$, it can often be replaced with either the maximum absolute value of the sum of the sound speed on each side of the interface and the local fluid velocity, or the average of these on both sides of the interface.
General symmetric fluxes like this are simple and efficient, but tend to be diffusive, smearing features across many grid cells, especially if the first-order Godunov or van Leer-style reconstruction methods are used.
This can be effectively combatted by the use of high-order reconstruction schemes and adaptive mesh refinement techniques, providing simple and robust solutions thanks to the reconstruction scheme.

\subsection{Time Integration, Stability, and Splitting Schemes}\label{Sec:HydroStability}

Explicit schemes are only stable (i.e. do not diverge or introduce spurious oscillations) if the Courant-Friedrichs-Lewy (CFL) condition is met.
The CFL condition states that the numerical domain of dependence of the the equation (i.e. the terms used in the computation of a value in the next timestep) must encompass the analytic domain of dependence to ensure that all necessary information is taken into account.
This is a necessary, but not sufficient condition, and the exact requirements of each scheme can often be derived analytically, although it is common to apply an additional safety margin to the maximum permitted value of the CFL condition.

For explicit methods the CFL condition sets the maximum timestep that can be used based on the current simulation conditions.
In a hyperbolic system the CFL condition takes the form
\begin{equation}
    C = \frac{\varv\Delta t}{\Delta x},
\end{equation}
and will be constrained to a maximum value ($\leq 1$) for stability of an explicit method.
Most implicit methods do not place an upper limit on the CFL condition for stability (as all points are coupled), but typically need to remain $\sim 1$ to avoid losing fine detail in the solution.
This is discussed \citet{Viallet2011}, who comment that CFL $\sim 1$ serves as an accuracy criterion and typically represents the optimum accuracy/computational cost ratio despite the method being nominally stable for large CFL values.

It is difficult to achieve better than second order accuracy due to the temporal discretisations discussed so far, but by viewing $F_i^t$ as the flux at time $t$, we can achieve higher order accuracy by applying a multi-step method to more accurately integrate these fluxes (and any associated source terms) over time.
A good choice for this is a method in the family of total-variation diminishing Runge-Kutta methods \citep[e.g.][]{Shu1988}.
These multi-step time integration methods can be combined with fractional step methods that allow us to perform \emph{operator splitting}.
That is, splitting an equation into two subproblems that can solved independently.
An example of this would be the radioactive decay of an isotope transported by advection.
Splitting this into an advection and a reaction problem allows for standard methods to be used in both of these problems, but clearly their results need to be coupled to each other.
A naive first approach is to solve one of the subproblems over the timestep, and then solve the other, but this method cannot be better than first order accurate in time for coupled subproblems.
A commonly used approach that is second order is known at Strang splitting \citep{Strang1968}, which consists of time-advancing the first subproblem by half the timestep, then the second by the whole timestep, before once again advancing the first subproblem by half the timestep.
This method can provide second-order accuracy.
Many more advanced splitting procedures have been developed, but Strang splitting remains widely used due to its ease of implementation.

\citet{LeVeque1997} comments that in many situations the first-order splitting described above performs better than would likely be expected from its formal first-order accuracy.
This is because the errors introduced are equivalent to solving the problem at a slightly different time, differing by up to a single timestep.
Whilst this renders the method first-order accurate, the quality of the solution is still primarily controlled by the quality of the methods used to solve the subproblems, and this exceedingly simple splitting scheme can often be applied with no problems.

There are many nuances to each of the techniques needed to accurately and robustly solve conservation laws numerically, and no single \emph{correct} method to use.
Indeed, formulations based on finite volume, finite difference, finite element, and discontinuous Galerkin methods are all in use in \Sota{} research codes.
The above is intended to serve as a \emph{somewhat opinionated} introduction to the complexities of these methods, but by no means present a conclusion on how conservation laws should be solved.
It is hoped that this introduction will improve general understanding of this aspect of RHD codes.

\section{Conduction}\label{Sec:NumericalConduction}

When expressed in terms of energy the heat equation in a plasma along a magnetic field line is given by
\begin{equation}
    \frac{\partial E}{\partial t} = \frac{\partial}{\partial z}\left( \kappa_0 T^{5/2} \frac{\partial T}{\partial z} \right),
\end{equation}
with spatial coordinate $z$, and coefficient $\kappa_0$ varying, but typically taken to be approximately \SI{1e-6}{\erg\per\centi\metre\per\second\per\kelvin\tothe{7/2}} \citep{Spitzer1953,Braginskii1965} based on deviations from a fully ionised hydrogen plasma.
The form of this equation poses several problems, the most significant being that in regions of sufficiently steep temperature gradients, the conductive flux approaches infinity.
Clearly this is not physical and there is a maximum limit, known as the free-streaming limit, at which all electrons in the plasma are flowing at their thermal speed with this heat gradient, representing a finite limit on this term.
As this free-streaming limit is approached the heat flux becomes non-local and depends on the global temperature and density structure in the loop \citep{Battaglia2009}.
\citet{Campbell1984} provides an expression for the transport coefficients used to determine the conductive flux through an ionised plasma and smoothly handles both the Spitzer-H\"{a}rm, locally limited, and non-locally limited regimes.
This approach can be applied in numerical simulations but more frequently (such as in RADYN) the method of \citet{FISHER1985} is applied which smoothly limits the local flux to remain less than the local free-streaming limit, helping to stabilise this equation.

Due to its parabolic nature, an explicit solution of the heat equation can be extremely costly; stability requirements provide a timestep requirement scaling with the inverse square of the grid spacing, rather than linearly as in the case of most hyperbolic equations.
Any attempt to explicitly integrate the heat equation on a timestep limit set by the hydrodynamic equations will likely be met with rapid divergence of any small perturbation in the data.
This renders the conduction term stiff compared to the hydrodynamical terms and it may therefore be advantageous to use an implicit method  to guarantee stability.
Several alternative methods have been developed, such as expressing the parabolic equation as a hyperbolic wave equation \citep{Rempel2016}, implicit-explicit methods (which may also be used for integrating stiff source terms such as the atomic level population transition rates) \citep[e.g.][]{Ascher1995}, or accepting the cost of an explicit discretisation in exchange for simplicity and accuracy \citep{Bradshaw2003, Bradshaw2013}.

Recently, discussion has emerged around the concept of turbulence suppressed conduction \citep{Bian2016}, where turbulence restricts the motion of electrons along the loop, and it has been suggested that the standard assumption of collisionally-dominated conduction may be a significant overestimation of true conduction rates.
Simulations using the zero-dimensional enthalpy based EBTEL code \citep{Klimchuk2008} currently suggest that this turbulent suppression alone is insufficient to maintain the high coronal temperatures and slow cooling times seen in observations, but likely represents an important component of this effect \citep{Bian2018}.
As part of further investigation, work is currently under way to integrate these effects in the \Radyn{} and HYDRAD codes.

\section{Discussions}

We have provided an overview of the commonly used techniques for simulating solar flares and the observable radiation produced from these events.
The core components of radiative transfer, hydrodynamics, and heat conduction have been discussed in depth, along with an overview of the numerical treatments of these terms.
Whilst there are other important terms in the RHD equations, such as the heating model, these represent the core of any flare simulation.

Modern RHD simulations have significantly enhanced our understanding of flares, but remain limited in the dimensionality of their treatments and cannot currently produce plausible synthetic light-curves without the artificial superposition of many individual simulations \citep[e.g.][]{Kerr2020}.
It is likely that significant progress will be made on these limitations in the coming 1--2 decades as computing power and numerical techniques improve.
In the meantime, we can undertake numerical experiments to evaluate individual components of these larger treatments (along with re-evaluating certain current assumptions) and we will present several of these using our \Lw{} based experiments.