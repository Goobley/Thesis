\chapter{Preface \& Acknowledgements}

It took me a long time to \emph{grok} the concepts of NLTE radiative transfer, probably due to hubris.
I never imagined, at the start of this endeavour, that it would come to represent the core of my research interests.
At some point after learning and playing with LTE methods, I convinced myself that the problem couldn't get much more difficult.
After all, how hard can it be to determine the opacity of a plasma?
The answer, it turns out, is ``quite hard'', but that's a topic for the rest of this thesis.
The methods we will discuss typically take a journey through a complex space to arrive at their destination --- there is an unfortunate lack of direct methods in this field.
I don't know whether I've arrived at my destination yet, but this is certainly a milestone, and there are many people I am grateful to for guiding me along this meandering path.

First and foremost, I would like to thank my supervisor, Lyndsay Fletcher, for letting me explore and follow the excitement gradient into the most interesting little valley.
In equal parts always ready to point me in the right direction, tear my bad ideas to shreds, and reassure me that I hadn't just wasted the last $n$ months.

There is no chance that I would be where I am today without the support and guidance of Paulo Sim\~{o}es who really took me under his wing, introduced me to many aspects of numerical modelling, and was always willing to spend time sounding out new ideas.
You really helped me learn how to be a scientist and engage with the community.

To my RT and inversion mentor, Ivan Mili\'{c}: Thank you!
Your enthusiasm is infectious, and your hospitality fantastic.
I'll never forget cycling around Boulder, or the ``quick questions'' that turned into chats that lasted for hours.
Thank you to everyone else at the National Solar Observatory for being so welcoming, full of different perspectives, and fun to be around, both in and out of the office.
Hazel and J{\o}rgen, thanks for providing me with a home in Boulder, rather than just a place to stay!

Denizens of the 604 office, I don't know half of you as well as I should like\dots{}
I know that you will all support each other as you finally get back in there.
Would that we had had more time in that little box of science: I have missed the whiteboard sessions, the impromptu chess, and the general banter.
John and Aaron, thanks for listening to my regular rants about numerical methods; I hope some of my Python advice has been useful!
To everyone else along the astronomy corridor, thank you for your passion and willingness to share the latest exciting discovery, scientific or otherwise, especially at 4\,pm coffee.

A personal thank you to Mats Carlsson for taking time to explain the intricacies of \Radyn{}, and nudge me in the directions of interesting problems.
Thanks also to Petr Heinzel and Jana Kašparová for listening to my ideas and then ensuring that I test them with sufficient rigour!

To the bois of FLaD, your entertaining company got me through the lockdowns with some amount of (in)sanity intact!
And where would I be without the soundscapes that have infused throughout this work?
Thanks to Devin Townsend, Eluveitie, Leprous, Myrkur, Babymetal, Nightwish, Wardruna, and Heilung\dots{} to name but a few.

I must of course thank my parents for always encouraging my curiosity, making me consider the world around me, and supporting all my endeavours.
You made me a person capable of handling this.

Throughout all the time spent at home, one person has stayed at my side (admittedly sleeping most of the time).
Thanks for the company Augustus, some cats have earned co-authorships for less.
Finally, Clara, none of this would have happened without you.
Thank you, pup, for all of your love, understanding, friendship, and encouragement.
Let's see where the next adventure takes us.