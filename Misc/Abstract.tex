\chapter{Abstract}

The chromosphere is a thin layer of the observable solar atmosphere maintained in a complex energy balance that is not well understood.
Chromospheric conditions can only be probed through observations of radiation, primarily atomic spectral lines.
The shapes and intensities of these spectral lines cannot be correctly reproduced by assuming local thermodynamic equilibrium (LTE) conditions, and are instead affected by non-local radiation fields.
During solar flares, an enormous amount of energy is deposited in the chromosphere, heating the plasma, provoking dramatic dynamic reactions and substantial changes in the observed line profiles.
A two-pronged approach is therefore required to develop our understanding of this region and its reaction to flares: detailed observations, and a theoretical framework derived from complex modelling by which to interpret them.

In this thesis we focus on techniques for modelling optical spectral lines in solar flare conditions, including a reassessment of previous assumptions, the radiative influence of a flare on neighbouring atmosphere, and present a novel machine learning inversion technique.
Our choice of optical lines is motivated by the exceptional resolution and cadence of modern and upcoming ground-based observatories, enabling a highly detailed exploration of this region.
All of the radiative modelling presented is performed with our new Python framework \Lw{}, which aims to facilitate the development of complex radiative transfer simulations.

In Chaps.~\ref{Chap:Intro}, \ref{Chap:FlareModelling}, and \ref{Chap:FlareObservations}, we introduce the necessary background material.
We first present a brief description of the outer layers of the solar atmosphere and general introduction to solar flares.
Then, in Chap.~\ref{Chap:FlareModelling}, we provide an overview of field-aligned radiation hydrodynamic modelling of solar flares, and associated numerical techniques, including an in-depth discussion of radiative transfer in non-LTE conditions.
In Chap.~\ref{Chap:FlareObservations}, we introduce the two primary spectral lines used in our research, \Ha{} and \CaLine{}, along with an overview of the inverse problem of radiative transfer (determining the atmospheric conditions responsible for an observed spectrum), and the concept of response functions, which describe the radiative response to a perturbation in atmospheric parameters.
We also provide a brief introduction to machine learning techniques.

The design philosophy and implementation of \Lw{} is described in Chap.~\ref{Chap:Lw}.
We also provide validation examples for the various features of this framework, presenting comparisons with \Sota{} radiative transfer codes RH and SNAPI, along with two further tests to validate its time-dependent treatment of atomic level populations.

In Chap.~\ref{Chap:TimeDepRt}, we apply \Lw{} to the synthesis of radiation from flare models produced by the radiation hydrodynamic code \Radyn{} and present three different case studies.
% This includes the implementation of advection in a manner compatible with \Radyn{}, and three different case studies.
Firstly, using two different \Radyn{} simulations, we investigate the photoionising effects of the hydrogen Lyman lines on \Caii{} and show that this has a significant effect on the emergent line profiles and a 10--15\,\% change in chromospheric radiative losses.
Secondly, the importance of a time-dependent treatment of \Caii{} is tested, with only minor deviations found due to ignoring these effects.
Finally, we present a modification of this tool to include the effects of partial frequency redistribution (PRD) and synthesise the radiation from a \Radyn{} model, despite the poor convergence of our technique.
The Doppler-like approximate PRD treatment used for the Lyman lines is found to remain relatively accurate in the flaring simulation, but the \Caii{} K line profile and radiative losses differ more significantly.
We also present a time-dependent formulation of response functions that, with further development, should allow for greater interpretability of spectral line formation in complex radiation hydrodynamic models.

The extension and application of \Lw{} to a two-dimensional slab of quiet Sun plasma adjacent to a \Radyn{} flare model is presented in Chap.~\ref{Chap:2DRT}.
This slab is held at a constant pre-flare temperature and density structure, whilst the atomic level populations and electron density are allowed to vary with the incoming radiation.
Significant enhancements in the \Ha{} and \CaLine{} line profiles are found in excess of \SI{1}{\mega\m} from the flare, although no continuum effects are seen in this simple model.
These enhancements found in this simple model are compared against observations from the CRisp Imaging SpectroPolarimeter (CRISP) on the Swedish Solar Telescope (SST) and effects of a similar order of magnitude are found, although substantial differences remain.

RADYNVERSION, described in Chap.~\ref{Chap:Radynversion}, is a novel deep learning based inversion technique built on an invertible neural network, allowing inference of chromospheric conditions multiple orders of magnitude faster than conventional techniques, outside of the constraints of statistical and hydrostatic equilibrium that are ill-suited to flares.
We discuss in detail the theory of its construction, and provide a brief proof of concept application to SST/CRISP data, which is found to agree with previous analysis of this event.