\chapter{Flare Observations}

\begin{itemize}
    \item How flares are observed.
    \item Important optical spectral lines \Ha{}, \CaLine{}; observations thereof.
    \item Forward Modelling
    \item Inversions
    \item Machine Learning
\end{itemize}

\section{Important Optical Spectral Lines}

Ground-based telescopes can be much larger and more complex than their space-based counterparts for an equivalent budget. In exchange for this they can only observe in limited bands (due to atmospheric effects), and suffer from the effects of atmospheric turbulence. With these telescopes we therefore focus on optical and near-infrared lines, which are the least affected.

We shall focus on two of the strongest optical lines, \Ha{} at \SI{656.3}{\nano\m}, and \CaLine{}. Both of these lines form in the chromosphere (albeit at different altitudes), and carry large amounts of information pertaining to the details of their formation.

\subsection{CRISP observations of these}
\subsection{DKIST, \& the future.}